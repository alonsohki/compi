\chapter{Puntos opcionales}

Hemos decidido llevar a cabo los 5 puntos optativos propuestos en la práctica. La descripción de cada uno de ellos, aparece debajo.

\section{Expresiones booleanas}
	
	\subsection{Disyunción, conjunción y negación de expresiones booleanas}
	
		Soportamos expresiones booleanas complejas mediante la técnica de cortocircuitos.
		
		Se ha tenido en cuenta la precedencia de operadores ($not << and << or$).

	\subsection{Tipo de datos booleano y expresión booleana}
	
		Se pueden crear variables que almacenan valores booleanos.
		
		Las expresiones booleanas son un tipo que representa un expresión condicional cuyos saltos aún no han sido completados.
		
	\subsection{Palabras reservadas true, false}
	
		Se ha incluido dos palabras reservadas para representar los valores booleanos.
		
	\subsection{Problemática en la implementación de tipos booleanos}
	
		La evaluación de un tipo como booleano, requiere la siguiente transformación a booleano. 
		
		\begin{itemize}
			\item \ter{Entero}: Falso cuando sea 0, True en cualquier otro caso.
			\item \ter{Real}: False cuando sea 0.0, True en cualquier otro caso.
			\item \ter{Expresion booleana}: Se añaden instrucciones asignando a la variable un valor booleano, y se completan los saltos de la expresión booleana, apuntando a estas nuevas instrucciones.
		\end{itemize}
	
	\subsection{Casos de prueba}
	
		\subsubsection*{Comprobamos el cortocircuitado, y la conversión a booleano}
		
		\lstinputlisting[basicstyle=\small]{../ejemplos/ejem_expresiones_booleanas.dat}
		\lstinputlisting[basicstyle=\small]{../ejemplos/ejem_expresiones_booleanas_salida.txt}

		\clearpage
		
		\subsubsection*{Comprobamos el operador not}		
		
		\lstinputlisting[basicstyle=\small]{../ejemplos/ejem_negaciones.dat}
		\lstinputlisting[basicstyle=\small]{../ejemplos/ejem_negaciones_salida.txt}
	
\clearpage


\section{Restricciones semánticas}

	\subsection{Comprobación del número, tipo y clase de los parámetros de las llamadas a subprogramas}
	
		Dicha comprobación tiene lugar en la regla \ter{parametros\_llamada}.
		
	\subsection{Conversión a tipos compatibles (Typecast)}
	
		En un lenguaje de programación hay conversiones de tipo implícitas en las llamadas a subprogramas, asignaciones e incluso dentro de las propias expresiones. Se han implementado las siguientes conversiones compatibles :
		
		Las filas representan el tipo de origen y las columnas representan el destino.
		
		\hspace{-2em}
		{
		\scriptsize
		\begin{tabular}{ | c |  c | c | c | c | c | c | c | c |} \hline
			~ 	& \ter{Entero} 	& \ter{Real}	& \ter{Booleano}	& \ter{ExpresiónBool} 	& \ter{Array}	& \ter{Funciones} 	& \ter{Procedimientos} 	& \ter{Unknown} \\ \hline
			
\ter{Entero} 	& \ding{52}		& \ding{52}	& \ding{52}		& \ding{52}				& -				& -					& -						& \ding{52}		\\ \hline
\ter{Real} 		& \ding{52}		& \ding{52}	& \ding{52}		& \ding{52}				& -				& -					& -						& \ding{52}		\\ \hline
\ter{Booleano}	& -				& -			& \ding{52}		& \ding{52}				& -				& -					& -						& \ding{52}		\\ \hline
\ter{ExpresiónBool} & -			& -			& \ding{52}		& \ding{52}				& -				& -					& -						& \ding{52}		\\ \hline
\ter{Array} 		& -				& -			& -				& -						& \ding{52}		& -					& -						& \ding{52}		\\ \hline
\ter{Funciones} 	& -				& -			& -				& -						& -				& -					& -						& \ding{52}		\\ \hline
\ter{Procedimientos} & -			& -			& -				& -						& -				& -					& -						& \ding{52}		\\ \hline
\ter{Unknown}	& \ding{52}		& \ding{52}	& \ding{52}		& \ding{52}				& \ding{52}		& \ding{52}			& \ding{52}				& \ding{52}		\\ \hline
		\end{tabular}
		}
	
	\subsection{Implementacion del Type cast}
	
		\lstinputlisting[basicstyle=\small]{type_cast.cpp}
		
	\subsection{Verificación en el acceso a arrays}
	
		Comprobamos que el identificador sea de tipo array, el número de indices sea correcto, y el tipo de estos sea entero o compatible con entero.
		
		Cuando se encuentra un array de un array, se concatenan internamente las dimensiones, pudiendo acceder al array de ambas formas.
		
		Si una función retorna un array, podemos realizar un acceso a sus elementos tras la llamada.
		
		La verificación de acceso a arrays tiene lugar en la regla \ter{acceso\_a\_array}.
		
	\subsection{Verificación en el acceso a subprogramas}
		
		Comprobamos que el identificador sea un subprograma, el número de parámetros sea correcto, el tipo de estos sea correcto o compatible y la clase\footnote{Una clase por referencia exige que el tipo sea exacto, sin admitir literales}.
	
		La verificación de acceso a arrays tiene lugar en la regla \ter{parametros\_llamada}.
		
	\subsection{Implementación de ámbitos de la tabla de símbolos}
	
		Comprobamos que un identificador haya sido declarado en el ámbito actual o uno de los padres.
		
	\subsection{Casos de prueba}
	
		\subsubsection*{Comprobamos el ámbito de salir si}		
		
		\lstinputlisting[basicstyle=\small]{../ejemplos/ejem_salir_si_fuera_de_bucle.dat}
		\lstinputlisting[basicstyle=\small]{../ejemplos/ejem_salir_si_fuera_de_bucle_salida.txt}

		\clearpage
		
		\subsubsection*{Comprobamos la redeclaración de variables}
		
		\lstinputlisting[basicstyle=\small]{../ejemplos/ejem_redeclare.dat}
		\lstinputlisting[basicstyle=\small]{../ejemplos/ejem_redeclare_salida.txt}

		\clearpage
		
		\subsubsection*{Comprobamos el typecast}
		
		\lstinputlisting[basicstyle=\small]{../ejemplos/tipos/typecast_asignacion.dat}
		\lstinputlisting[basicstyle=\small]{../ejemplos/tipos/typecast_asignacion_salida.txt}


\clearpage

\section{Llamadas a procedimientos y funciones}

	\subsection{Implementación}
	
		Se ha implementado la llamada, el paso de parámetros, y el retorno de funciones.

	\subsection{Diferenciación entre funciones y procedimientos}
	
		Un procedimiento no puede ser usado dentro de una expresión.
	
	\subsection{Casos de prueba}
		
		\subsubsection*{Comprobamos las llamadas a subprogramas}
		
		\lstinputlisting[basicstyle=\small]{../ejemplos/funciones/ejem_accesos_llamadas.dat}
		\lstinputlisting[basicstyle=\small]{../ejemplos/funciones/ejem_accesos_llamadas_salida.txt}
		
		\clearpage
		
		\subsubsection*{Comprobamos el acceso a arrays retonados por funciones}
		
		\lstinputlisting[basicstyle=\small]{../ejemplos/arrays/funcion_array.dat}
		\lstinputlisting[basicstyle=\small]{../ejemplos/arrays/funcion_array_salida.txt}
		
\clearpage

\section{Errores sintácticos}

	Se ha implementado el modo de pánico con distintas estrategias :
	
	\begin{enumerate}
	
		\item \ter{Fail} : La compilación se aborta.
		
		\item \ter{Ignore} : Se emula que el match hubiera sido certero.
		
		\item \ter{TryAgain} : Se igual que el anterior, pero manteniendo el \emph{Lookahead}.
		
		\item \ter{FindIt} : Avanza el \emph{Lookahead} hasta encontrar el token solicitado.
		
		\item \ter{FindNext} : Avanza el \emph{Lookahead} hasta encontrar uno de los siguientes de la regla que se estaba ejecutando.
		
	\end{enumerate}

		\subsubsection*{Comprobamos el panic mode}
		
		\lstinputlisting[basicstyle=\small]{../ejemplos/ejem_panicmode.dat}
		\lstinputlisting[basicstyle=\small]{../ejemplos/ejem_panicmode_salida.txt}

\clearpage

\section{Arrays multidimensionales}

	\subsection{Dominio del índice}
	
		Se ha considerado el cero como primer elemento del array.
		
	\subsection{Resolución de arrays de arrays}
	
		Se compactan los arrays, concatenando las dimensiones.
		
	\subsection{Acceso arrays}
	
		Se permite acceder a un array tanto especificando las dimensiones en una lista separada por comas, como con múltiples indirecciones.
	
	\subsection{Casos de prueba}
	
		\subsubsection*{Comprobamos los distintos modos de acceso a array}
		
		\lstinputlisting[basicstyle=\small]{../ejemplos/arrays/acceso_fragmentado.dat}
		\lstinputlisting[basicstyle=\small]{../ejemplos/arrays/acceso_fragmentado_salida.txt}
		