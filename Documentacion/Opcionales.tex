\chapter{Puntos opcionales}

\section{Expresiones booleanas}

● Ampliación del ETDS para recoger la traducción de expresiones booleanas. 
● Ampliación del traductor para implementar lo anterior. 

\section{Uso correcto de identificadores}


	Añadimos las instrucciones ST_PUSH y ST_POP que implementan el ámbito de la tabla de símbolos, durante la declaración del subprograma.
	
	\small
	\begin{tabular}{r c p{.72\textwidth}}
	decl\_de\_procedimiento 	& $\longrightarrow$ 	& \sem{ ST\_PUSH(); } \\
							&					& cabecera\_procedimiento declaraciones \\
							&					& \ter{comienzo} lista\_de\_sentencias\_prima \ter{fin} \\
							&					& \sem{ ADD\_INST(``finproc''); } \\
							&					& \ter{;} \\
							&					& \sem{ ST\_POP(); } \\
	\end{tabular}	
							
	
	Estrategia de tipos :

Cuando nos encontramos con 2 expresiones, intentamos convertilo al grupo más general :

Ejemplo : 

x = 5 * 1.3 + true;

1º)  convierte 5 a 5.0 mediante 5 * 1.0




Tratamiento estático y tratamiento dinámico :
	
● Definición de restricciones semánticas sobre uso correcto de identificadores. 
● Ampliación del ETDS para comprobar la corrección semántica. 
● Especificación funcional 
   de la Tabla de Símbolos. 
● Selección de una representación para la Tabla de Símbolos. 
● Implementación y prueba de la parte de análisis semántico. 

\section{llamadas a procedimientos}
● Ampliación de la gramática y del ETDS para permitir llamadas a procedimientos 
● Implementación y prueba de las llamadas a procedimientos.

\section{Errores sintácticos}
 
● Diseño del tratamiento de errores sintácticos 
● Implementación del tratamiento de errores sintácticos 

\section{Arrays multidimensionales}

\subsection{Ampliación de la ETDS}

\small
\begin{tabular}{r c p{.72\textwidth}}
	tipo 			& $\longrightarrow$	& \ter{array} \ter{[} lista\_de\_enteros \ter{]} \ter{de} tipo \\
					&					& \sem{ tipo.tipo := NEW\_ARRAY\_TYPE(lista\_de\_enteros, tipo); } \\
	
	\espacio
	
\end{tabular}
					
● Ampliación de la gramática y del ETDS para permitir el tipo array muldimensional. 
● Implementación y prueba de los arrays. 
