\chapter{Puntos opcionales}

\section{Expresiones booleanas}

● Ampliación del ETDS para recoger la traducción de expresiones booleanas. 
● Ampliación del traductor para implementar lo anterior.

Añadimos las palabras reservadas \ter{ true } y ter{ false } como constantes booleanas, y añadimos un nuevo tipo de datos \ter{ booleano }. Además,
definimos un nuevo tipo de ETDS para expresiones, que además de las ya vistas expresiones numnéricas, pueden contener expresiones booleanas;
así como unos operadores para ellas \ter { or }, \ter{ and } y \ter{ not}.

\small
\begin{tabular}{r c p{.72\textwidth}}
tipo                                             	& $\longrightarrow$                     & \ter{ booleano } \sem{ tipo.tipo = NEW_BASIC_TYPE(REAL); } \\
\espacio
\end{tabular}

\begin{tabular}{r c p{.72\textwidth}}
booleano                                           	& $\longrightarrow$                     & \ter{ true } | \ter{ false } \\
\espacio
\end{tabular}

\begin{tabular}{r c p{.72\textwidth}}
expresiones                                           	& $\longrightarrow$                     & \ter{ = } expresion \\
expresion                                           	& $\longrightarrow$                     & disyuncion \\
disyuncion                                          	& $\longrightarrow$                     & conjuncion disyuncion' \\
disyuncion'                                          	& $\longrightarrow$                     & \ter{ or } conjuncion disyuncion' \\
                                                        &                                       & | \xi
conjuncion                                          	& $\longrightarrow$                     & relacional conjuncion' \\
conjuncion'                                          	& $\longrightarrow$                     & \ter{ and } relacional conjuncion' \\
                                                        &                                       & | \xi
relacional                                          	& $\longrightarrow$                     & aritmetica relacional' \\
relacional'                                          	& $\longrightarrow$                     & \ter{ oprel } aritmetica relacional' \\
                                                        &                                       & | \xi
aritmetica                                         	& $\longrightarrow$                     & termino aritmetica' \\
aritmetica'                                         	& $\longrightarrow$                     & \ter{ opl2 } termino aritmetica' \\
                                                        &                                       & | \xi
termino                                         	& $\longrightarrow$                     & negacion termino' \\
termino'                                         	& $\longrightarrow$                     & \ter{ opl1 } negacion termino' \\
                                                        &                                       & | \xi
negacion                                         	& $\longrightarrow$                     & \ter{ not } factor \\
                                                        &                                       & | factor \\
factor                                           	& $\longrightarrow$                     & \ter{ - } factor' \\
                                                       	&                                       & factor' \\
factor'                                         	& $\longrightarrow$                     & \ter{ ID } array\_o\_llamada \\
                                                        &                                       & | \ter{ INTEGER }
                                                        &                                       & | \ter{ REAL }
                                                        &                                       & | booleano
                                                        &                                       & \ter{ ( } expresion \ter{ ) }
opl1                                            	& $\longrightarrow$                     & \ter{ * } | \ter{ / } \\
opl2                                            	& $\longrightarrow$                     & \ter{ + } | \ter{ - } \\
oprel                                            	& $\longrightarrow$                     & \ter{ $>$ } \\
                                                        &                                       & | \ter{ $<$ } \\
                                                        &                                       & | \ter{ $\leq$ } \\
                                                        &                                       & | \ter{ $\geq$ }
                                                        &                                       & | \ter{ $==$ }
                                                        &                                       & | \ter{ $/=$ }
\espacio
\end{tabular}

\section{Uso correcto de identificadores}


	Añadimos las instrucciones ST_PUSH y ST_POP que implementan el ámbito de la tabla de símbolos, durante la declaración del subprograma.
	
	\small
	\begin{tabular}{r c p{.72\textwidth}}
	decl\_de\_procedimiento 	& $\longrightarrow$ 	& \sem{ ST\_PUSH(); } \\
							&					& cabecera\_procedimiento declaraciones \\
							&					& \ter{comienzo} lista\_de\_sentencias\_prima \ter{fin} \\
							&					& \sem{ ADD\_INST(``finproc''); } \\
							&					& \ter{;} \\
							&					& \sem{ ST\_POP(); } \\
	\end{tabular}	
							
	
	Estrategia de tipos :

Cuando nos encontramos con 2 expresiones, intentamos convertirlo al grupo más general :

Ejemplo : 

x = 5 * 1.3 + true;

1º)  convierte 5 a 5.0 mediante 5 * 1.0




Tratamiento estático y tratamiento dinámico :
	
● Definición de restricciones semánticas sobre uso correcto de identificadores. 
● Ampliación del ETDS para comprobar la corrección semántica. 
● Especificación funcional 
   de la Tabla de Símbolos. 
● Selección de una representación para la Tabla de Símbolos. 
● Implementación y prueba de la parte de análisis semántico. 

\section{Llamadas a procedimientos}
● Ampliación de la gramática y del ETDS para permitir llamadas a procedimientos 
● Implementación y prueba de las llamadas a procedimientos.

En nuestro caso, hemos permitido llamadas a procedimientos y funciones.

\subsection{Ampliación de la ETDS}

\small
\begin{tabular}{r c p{.72\textwidth}}
        decl\_de\_subprogs              & $\longrightarrow$     & decl\_de\_procedimiento decl\_de\_subprogs \\
                                        & $\longrightarrow$     & | decl\_de\_funcion decl\_de\_subprogs \\
                                        &                       & | \xi
        decl\_de\_procedimiento		& $\longrightarrow$	& cabecera\_procedimiento declaraciones \ter{ comienzo } lista\_de \_sentencias \ter{ fin } \ter{ ; } \\
        decl\_de\_funcion 		& $\longrightarrow$	& cabecera\_funcion declaraciones lista\_de \_sentencias' \ter{ fin } \ter{ ; } \\
        cabecera\_procedimiento		& $\longrightarrow$	& \ter{ procedimiento } \ter{ ID } argumentos \\
        cabecera\_funcion 		& $\longrightarrow$	& \ter{ funcion } \ter{ ID } argumentos \ter{ retorna } tipo \\
	argumentos                      & $\longrightarrow$	& \ter{ \( } lista\_de\_param \ter{ \) } \\
                                        &                       & | \xi
	lista\_de\_param		& $\longrightarrow$     & lista\_ident \ter{ : } clase\_param tipo resto\_lis\_de\_param  \\
        resto\_lis\_de\_param		& $\longrightarrow$     & \ter{ ; } lista\_ident \ter{ : } clase\_param tipo resto\_lis\_de\_param  \\
                                        &                       & | \xi
        lista\_de\_sentencias'          & $\longrightarrow$     & lista\_de\_sentencias

	\espacio

\end{tabular}

Se añade también el cambio realizado en la regla inicial (para permitir hacer procedimientos o funciones):

\small
\begin{tabular}{r c p{.72\textwidth}}
        programa                        & $\longrightarrow$     & \ter{ programa ID } \\
					&					& \sem{ ADD\_INST(``prog'' || ID.value); } \\
                                        &                       & declaraciones decl\_de\_subprogs \\
                                        &                       & \ter{ comienzo } lista\_de\_sentencias' \ter{ fin ; } \\
					&					& \sem{ ADD\_INST(``halt''); } \\

	\espacio

\end{tabular}

\section{Errores sintácticos}
 
● Diseño del tratamiento de errores sintácticos 
● Implementación del tratamiento de errores sintácticos 

\section{Arrays multidimensionales}

● Ampliación de la gramática y del ETDS para permitir el tipo array muldimensional.
● Implementación y prueba de los arrays.

\subsection{Ampliación de la ETDS}

\small
\begin{tabular}{r c p{.72\textwidth}}
	tipo 			& $\longrightarrow$	& \ter{array} \ter{[} lista\_de\_enteros \ter{]} \ter{de} tipo \\
					&					& \sem{ tipo.tipo := NEW\_ARRAY\_TYPE(lista\_de\_enteros, tipo); } \\
	lista\_de\_enteros      & $\longrightarrow$     & \ter{ INTEGER } resto\_lista\_enteros \\
                                                                                & \sem{ lista\_enteros.ints = JOIN(INIT\_LIST(INTEGER.value), resto\_lista\_enteros.ints); } \\
        resto\_lista\_enteros   & $\longrightarrow$     & | \ter{ , } \ter{ INTEGER } resto\_lista\_enteros \\
                                                                                & \sem{ resto\_lista\_enteros.ints = JOIN(INIT\_LIST(INTEGER.value), resto\_lista\_enteros.ints); } \\
                                &                       & | \xi \sem{ resto\_lista\_enteros.ints = EMPTY\_LIST(); }
	\espacio
	
\end{tabular}
