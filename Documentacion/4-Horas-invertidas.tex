
\chapter{Horas invertidas}

    \section{Jon Ander Hernández}
    
        \begin{itemize}
            \item 1 hora en el estudio del enunciado.
            \item 2 horas en el estudio y diseño de los autómatas.
            \item 1 hora en la creación de los dibujos de los autómatas.
            \item 1 hora en el esqueleto de la documentación en \LaTeX.
            \item 4 horas de Extreme Programming (en Pair Programming) en la implementación.
         \end{itemize}
    
    \section{Alberto Alonso}
    
        \begin{itemize}
            \item 1 hora en estudio del enunciado y diseño de algoritmos.
            \item 30 minutos en los autómatas.
            \item 1 hora en el esqueleto de la aplicación.
            \item 15 minutos en la puesta a punto del subversion.
            \item 1 hora de implementación.
            \item 5 horas de Extreme Programming (en Pair Programming) en la implementación.
			\item 10 horas en la ETDS básica.
			\item 11 horas en la recopilación de información e implementación de la estructura base del analizador sintáctico/semántico.
			\item 9 horas en la programación de la información de tipos y la tabla de símbolos.
			\item 24 horas con la ETDS final.
			\item 8 horas con abstracciones funcionales, funciones de cadenas de texto y arreglos varios.
			\item 2.5 horas implementando el modo de pánico.
        \end{itemize}
    
    \section{Gorka Blanco Gutierrez}
    
        \begin{itemize}
            \item 1 hora en estudio del enunciado y diseño de algoritmos.
            \item 3 horas estudio y limpieza del antiguo código en C y Java.
            \item 1 hora de documentación en \LaTeX.
            \item 2 horas de Pair Programming.
        \end{itemize}