
\chapter{Abstracciones funcionales}

Lista de abstracciones utilizadas, detallando los parámetros con sus tipos y una breve 
descripción :

\section{Control de código}

\begin{enumerate}
	\item \ter{ADD\_INST} ( instrucción ) \\
			Añade la instrucción al resultado, e incrementa la posición actual en el código intermedio.
			
	\item \ter{GET\_REF} ( ) \\
			Obtiene una referencia a la posición actual en el código intermedio.
	
	\item \ter{NEW\_INDENT} ( ) \\
			Crea un nuevo identificador temporal.

	\item \ter{COMPLETE} ( lista, ref ) \\
			Completa las instrucciones contenidas en lista, añadiéndoles la referencia ref.
\end{enumerate}

\section{Listas}

\begin{enumerate}
	\item \ter{EMPTY\_LIST} () \\
			Crea una lista vacia.
			
	\item \ter{INIT\_LIST} ( elemento ) \\
			Crea una lista y la inicializa con un elemento.
			
	\item \ter{JOIN} ( lista$_{1}$, lista$_{2}$ ) \\
			Une dos listas.
			
	\item \ter{LIST\_SIZE} ( lista ) \\
			Retorna el tamaño de la lista.
			
	\item \ter{LIST\_ITEM} ( lista, n ) \\
			Devuelve el enésimo elemento de la lista.
	
	\item \ter{FOREACH} ( lista \ter{AS} iterador ) \\
			Recorre la lista especificada como argumento en el parámetro iterador.
	
\end{enumerate}

\section{Tipos}

\begin{enumerate}

	\item \ter{NEW\_BASIC\_TYPE} ( tipo ) \\
			Genera la representación en memoria de un tipo.

	\item \ter{IS\_INTEGER} ( attr\_tipo ) \\
			Comprueba si el parámetro representa a un entero.
			
	\item \ter{IS\_REAL} ( attr\_tipo ) \\
			Comprueba si el parámetro representa a un real.

	\item \ter{IS\_NUMERIC} ( attr\_tipo ) \\
			Comprueba si el parámetro representa a un tipo numérico (entero o real).

	\item \ter{IS\_BOOLEAN} ( attr\_tipo ) \\
			Comprueba si el parámetro representa a un booleano.

	\item \ter{IS\_BOOLEANEXPR} ( attr\_tipo ) \\
			Comprueba si el parámetro representa a una expresión condicional.

	\item \ter{IS\_ARRAY} ( attr\_tipo ) \\
			Comprueba si el parámetro representa a un array.

	\item \ter{IS\_PROCEDURE} ( attr\_tipo ) \\
			Comprueba si el parámetro representa a un procedimiento.

	\item \ter{IS\_FUNCTION} ( attr\_tipo ) \\
			Comprueba si el parámetro representa a una función.

	\item \ter{IS\_UNKNOWN} ( attr\_tipo ) \\
			Comprueba si el parámetro representa a un tipo desconocido.
		
	\item \ter{TYPE\_OF} ( attr\_tipo ) \\
			Retorna una representación literal de un tipo.
			
	\item \ter{TYPE\_CAST} ( nombre, tipo\_origen, tipo\_destino ) \\
		  \ter{TYPE\_CAST} ( nombre, tipo\_origen, tipo\_destino, lista\_true, lista\_false ) \\	
			Transforma la variable nombre del tipo origen al tipo destino, cuando sea posible o necesaria la conversión.

\end{enumerate}

\section{Arrays}

\begin{enumerate}

	\item \ter{NEW\_ARRAY\_TYPE} ( tipo )
			Crea la representación en memoria del tipo de un array, que contiene tipo.
			
	\item \ter{ARRAY\_CONTENT} ( array )
			Retorna el tipo del array.
			
	\item \ter{ARRAY\_SIZE} ( array )
			Retorna el tamaño total del array.
			
	\item \ter{ARRAY\_DEPTH} ( array )
			Retorna el número de dimensiones del array.
	
	\item \ter{ARRAY\_DIMENSION} ( array, n )
			Retorna la longitud de la enésima dimensión del array.

\end{enumerate}

\section{Funciones y procedimientos}

\begin{enumerate}
		
	\item \ter{NEW\_PROCEDURE\_TYPE} ( nombre, lista\_parametros ) \\
			Crea la representación en memoria del tipo de un procedimiento.
			
	\item \ter{NEW\_FUNCTION\_TYPE} ( nombre, lista\_parametros, tipo\_retorno )
			Crea la representación en memoria del tipo de una función.
			
	\item \ter{SUBPROG\_NUM\_PARAMS} ( tipo\_subprog ) \\
			Retorna el número de parámetros del subprograma.

	\item \ter{SUBPROG\_PARAM} ( tipo\_subprog, n ) \\
			Retorna el tipo del enésimo parámetro del subprograma.

	\item \ter{SUBPROG\_PARAM\_CLASS} ( tipo\_subprog, n ) \\
			Retorna la clase del enésimo parámetro del subprograma.
	
	\item \ter{FUNCTION\_RETURN} ( tipo\_subprog, n ) \\
			Retorna el tipo de retorno de la función.
			
\end{enumerate}

\section{Tabla de símbolos}

\begin{enumerate}

	\item \ter{ST\_PUSH} ( ) \\
			Apila un nuevo ámbito en la tabla de símbolos.
			
	\item \ter{ST\_POP} ( ) \\
			Desapila el ámbito de la cima de la tabla de símbolos.

	\item \ter{ST\_ADD} ( nombre, tipo ) \\
			Registra el identificador nombre con el tipo dado en la tabla de símbolos.\\
			En caso de que ya exista un símbolo con ese nombre, se llevará a cabo una de las siguientes políticas de reemplazo :
			%
				\begin{enumerate}
				
					\item \ter{Fail}: La compilación se aborta.
					
					\item \ter{Replace}: Se reemplaza la antigua declaración con la nueva. \\
						  \emph{Política por defecto}.
					
					\item \ter{Ignore}: Se mantiene la antigua declaración.
					
				\end{enumerate}
			%
			Retorna true, en caso de que se permita realizar la declaración.
					
	\item \ter{ST\_EXISTS} ( nombre ) \\
			Retorna true en caso de existir un símbolo con ese nombre.
				
	\item \ter{ST\_GET\_TYPE} ( nombre ) \\
			Retorna el tipo del símbolo con el nombre dado.
	
\end{enumerate}

\section{Control de errores}

\begin{enumerate}
	\item \ter{WARNING} ( mensaje ) \\
			Muestra un mensaje de advertencia.
			
	\item \ter{ERROR} ( mensaje ) \\
			Muestra un mensaje de error, y cancela la generación de código.
	
	\item \ter{DIE} ( mensaje ) \\
			Muestra un mensaje de error, y cancela la compilación.
\end{enumerate}
