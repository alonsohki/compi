\chapter{Implementación}


	La práctica se ha implementado mediante un Pseudolenguaje basado en macros de preprocesador de C++.
	
	Disponemos de dos macros, DECLARE\_RULE y DEFINE\_RULE que construyen las clases que implementan cada una de las reglas.
	
	\begin{description}
	
		\item[DECLARE\_RULE] ~ \\
			Mediante esta macro definimos el nombre y los atributos que dispondrá la clase que implementa la regla.
			
		\item[DEFINE\_RULE] ~\\
			Mediante esta macro construimos la clase, y completamos el método \emph{Execute()} encargado de hacer el análisis.
			
			En esta MACRO también definimos la lista de primeros y siguientes, para implementar el modo pánico y el branching.
			
		\item[MATCH] ~\\
			Realiza un Match, y activa el modo pánico en caso de fallo.
			
		\item[RULE] ~\\
			Apila una regla para ser ejecutada, y opcionalmente lo deja en una variable que usamos para modificar y obtener sus atributos (heredados y sintetizados).
			
			
	\end{description}
	
	
	\subsection{Ejemplo de expansión de Macros}
	
		\lstinputlisting[basicstyle=\small]{expansion_macros.cpp}