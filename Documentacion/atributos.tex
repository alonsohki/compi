\chapter{Atributos}

Lista de atributos, detallando para cada uno una descripción y la naturaleza del atributo (léxico, sintetizado o heredado) :

\begin{enumerate}
	\item \ter{L} = léxico 
	\item \ter{S} = sintetizado 
	\item \ter{H} = heredado 
\end{enumerate}

\section*{Listado de atributos}

\begin{tabularx}{\textwidth}{| r | c | c | X |} \hline

	\ter{No terminal}	& \ter{Tipo}		& \ter{Nombre}	& \ter{Descripcion} \\ \hline \hline
	
	%-----
	
	\ter{Todos los tipos de token} & \ter{L} 	& value			& El valor literal del token encontrado. \\ \hline
		
	programa 			&&& \\ \hline
	
	declaraciones 		&&& \\ \hline
	
	tipo 				& \ter{S} 		& tipo 			& El nombre interno del tipo especificado en la sintaxis. \\ \hline
	
	lista\_de\_ident 	& \ter{S} 		& ids			& Lista que contiene los identificadores encontrados. \\ \hline
	
	resto\_lista\_id 	& \ter{S} 		& ids			& Lista que contiene los identificadores encontrados. \\ \hline

	decl\_de\_subprogs 	&&& \\ \hline
	
	decl\_de\_subprograma &&& \\ \hline

	cabecera 			&&& \\ \hline
	
	argumentos 			&&& \\ \hline
	
	lista\_de\_param 	&&& \\ \hline
	
	clase\_par 			& \ter{S} 		& clase			& La clase de parámetro que se ha derivado de la sintaxis (referencia o valor). \\ \hline

	clase\_par’			& \ter{S} 		& clase			& La clase de parámetro que se ha derivado de la sintaxis (referencia o valor). \\ \hline
	
	lista\_de\_sentencias’ &&& \\ \hline
	
\end{tabularx}

\vfill

\clearpage

\begin{tabularx}{\textwidth}{| r | c | c | X |} \hline

	\ter{No terminal}	& \ter{Tipo}		& \ter{Nombre}	& \ter{Descripcion} \\ \hline \hline	
	
	%-----

	\multirow{2}{*}{lista\_de\_sentencias}  
						& \ter{H} 		& hinloop 		& Marcador que indica si estamos dentro de un bucle. \\ \cline{2-4} 
						& \ter{S} 		& salir\_si		& Lista de referencias a las instrucciones "salir\_si" contenidas dentro del ámbito de lista de sentencias. \\ \hline
						
	sentencia 			& \ter{S} 		& salir\_si		& Lista de referencias a las instrucciones "salir\_si" contenidas dentro de la sentencia. \\ \hline

	variable				& \ter{S} 		& nombre			& El nombre del identificador que define esta variable. \\ \hline

	\multirow{3}{*}{expresión} 
						& \ter{S} 		& nombre			& Nombre de la variable del programa o temporal que contiene el
														  valor de evaluar la expresión.  \\ \cline{2-4} 
						& \ter{S} 		& true			& Lista de referencias a saltos que han de completarse para definir a 
														  dónde saltará el programa si la \emph{evaluación de la expresión es cierta}. \\ \cline{2-4} 
						& \ter{S} 		& false			& Lista de referencias a saltos que han de completarse para definir a 
														dónde saltará el programa si la \emph{evaluación de la expresión es falsa}. \\ \hline

	opl2					& \ter{S}		& nombre			& Contiene el nombre del \emph{operador aritmético} (+ o -) \\ \hline

	opl1					& \ter{S}		& nombre			& Contiene el nombre del \emph{operador aritmético} (* o /) \\ \hline

	oprel 				& \ter{S}		& nombre			& Contiene el nombre del \emph{operador relacional} (+ o -) \\ \hline
	
	expresión\_simple	& \ter{S} 		& nombre			& El nombre de la variable del programa o temporal que contiene el 
														  valor de evaluar la expresión. \\ \hline
														  
	expresión\_simple’	& \ter{H} 		& hnombre		& Atributo utilizado para que los operadores binarios conozcan el 
														  elemento de su izquierda, o en caso de ser unario o un átomo se propague de 
														  nuevo hacia arriba el nombre original. \\ \cline{2-4} 
						
						& \ter{S} 		& nombre			& Nombre de la variable del programa o temporal que contiene el 
														  valor de evaluar la expresión. \\ \hline
														  
	término				& \ter{S} 		& nombre			& Nombre de la variable del programa o temporal que contiene el 
														  valor de evaluar la expresión. \\ \hline
														  
	término’				& \ter{H} 		& hnombre		& Atributo utilizado para que los operadores binarios conozcan el 
														  elemento de su izquierda, o en caso de ser unario o un átomo se propague de 
														  nuevo hacia arriba el nombre original. \\ \cline{2-4} 
														  
						& \ter{S} 		& nombre			& Nombre de la variable del programa o temporal que contiene el \\ \hline
	
	factor				& \ter{S} 		& nombre			& Nombre de la variable del programa o temporal que contiene el 
														  valor de evaluar la expresión. \\ \hline

	M					& \ter{S} 		& ref			& Referencia a la instrucción que se introducirá a continuación. \\ \hline
	
\end{tabularx}