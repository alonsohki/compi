\chapter{Atributos}

Lista de atributos, detallando para cada uno una descripción y la naturaleza del atributo (léxico, sintetizado o heredado) :

\begin{enumerate}
	\item \ter{L} = léxico 
	\item \ter{S} = sintetizado 
	\item \ter{H} = heredado 
\end{enumerate}

\section*{Listado de atributos}

\begin{tabularx}{\textwidth}{| r | c | c | X |} \hline

	\ter{No terminal}	& \ter{Tipo}		& \ter{Nombre}	& \ter{Descripcion} \\ \hline \hline
	
	%-----
	
	\ter{Todos los tipos de token} & \ter{L} 	& value			& El valor literal del token encontrado. \\ \hline
		
	programa 			&&& \\ \hline
	
	declaraciones 		&&& \\ \hline
	
	lista\_de\_ident 	& \ter{S} 		& ids			& Lista que contiene los identificadores encontrados. \\ \hline
	
	resto\_lista\_id 	& \ter{S} 		& ids			& Lista que contiene los identificadores encontrados. \\ \hline
	
	tipo 				& \ter{S} 		& tipo 			& El nombre interno del tipo especificado en la sintaxis. \\ \hline

	lista\_de\_enteros	& \ter{S}		& ints			& Lista de enteros. \\ \hline
	
	resto\_lista\_enteros & \ter{S}		& ints			& Lista de enteros. \\ \hline
	
	decl\_de\_subprogs 	&&& \\ \hline
	
	decl\_de\_procedimiento 	& \ter{S}		& nombre			& Identificador del procedimiento. \\ \cline{2-4}
							& \ter{S}		& args			& Lista de tipos de los parámetros. \\ \cline{2-4}
							& \ter{S}		& classes		& Lista de clases de los parámetros. \\ \hline
\end{tabularx}

\vfill

\begin{tabularx}{\textwidth}{| r | c | c | X |} \hline

	\ter{No terminal}	& \ter{Tipo}		& \ter{Nombre}	& \ter{Descripcion} \\ \hline \hline	
	
	%-----

	decl\_de\_funcion	 	& \ter{S}		& nombre			& Identificador de la función. \\ \cline{2-4}
							& \ter{S}		& args			& Lista de tipos de los parámetros. \\ \cline{2-4}
							& \ter{S}		& classes		& Lista de clases de los parámetros. \\ \cline{2-4}
							& \ter{S}		& tipoRetorno	& Tipo de retorno. \\ \hline
							
	cabecera\_procedimiento 	& \ter{S}		& nombre			& Identificador del procedimiento. \\ \cline{2-4}
							& \ter{S}		& args			& Lista de tipos de los parámetros.\\ \cline{2-4}
							& \ter{S}		& classes		& Lista de clases de los parámetros.\\ \hline
							
	cabecera\_funcion	 	& \ter{S}		& nombre			& Identificador de la función. \\ \cline{2-4}
							& \ter{S}		& args			& Lista de tipos de los parámetros.\\ \cline{2-4}
							& \ter{S}		& classes		& Lista de clases de los parámetros.\\ \cline{2-4}
							& \ter{S}		& tipoRetorno	& Tipo de retorno. \\ \hline
							
	argumentos 				& \ter{S}		& args			& Lista de tipos de los parámetros. \\ \cline{2-4}
							& \ter{S}		& classes		& Lista de clases de los parámetros. \\ \hline
						
	lista\_de\_param 		& \ter{S}		& args			& Lista de tipos de los parámetros. \\ \cline{2-4}
							& \ter{S}		& classes		& Lista de clases de los parámetros. \\ \hline
						
	resto\_lis\_de\_param 	& \ter{S}		& args			& Lista de tipos de los parámetros. \\ \cline{2-4}
							& \ter{S}		& classes		& Lista de clases de los parámetros. \\ \hline
	
	clase\_param				& \ter{S} 		& clase			& La clase de parámetro que se ha derivado de la sintaxis (referencia o valor). \\ \hline

	clase\_param’			& \ter{S} 		& clase			& La clase de parámetro que se ha derivado de la sintaxis (referencia o valor). \\ \hline
	
	lista\_de\_sentencias’ 	&&& \\ \hline

	lista\_de\_sentencias	& \ter{H}		& hinloop 		& Marcador que indica si estamos dentro de un bucle. \\ \cline{2-4} 
							& \ter{S}		& salir\_si		& Lista de referencias a las instrucciones "salir\_si" contenidas dentro del ámbito de lista de sentencias. \\ \hline
						
	sentencia 				& \ter{H}		& hinloop 		& Marcador que indica si estamos dentro de un bucle. \\ \cline{2-4}
							& \ter{S} 		& salir\_si		& Lista de referencias a las instrucciones "salir\_si" contenidas dentro de la sentencia. \\ \hline

	id\_o\_array				& \ter{H}		& hident			& El identificador. \\ \cline{2-4}
							& \ter{S}		& nombre			& El nombre de acceso a una variable o un array. \\ \hline
							
	asignacion\_o\_llamada	& \ter{H}		& hident			& El identificador. \\ \hline
	
	acceso\_a\_array			& \ter{H}		& hident			& El indentificador del array. \\ \cline{2-4}
							& \ter{H}		& htipo			& El tipo del array. \\ \cline{2-4}
							& \ter{S}		& offset			& El offset calculado \\ \cline{2-4}
							& \ter{S}		& tipo			& El tipo obtenido en el offset. \\ \hline

\end{tabularx}

\vfill

\begin{tabularx}{\textwidth}{| r | c | c | X |} \hline

	\ter{No terminal}	& \ter{Tipo}		& \ter{Nombre}	& \ter{Descripcion} \\ \hline \hline	
	
	%-----
	
	acceso\_a\_array'		& \ter{S}		& exprs			& Lista de identificadores de los subíndices del array. \\ \cline{2-4}
							& \ter{S}		& tipos			& Lista de tipos de los subíndices del array. \\ \hline
							
	parametros\_llamadas		& \ter{H}		& hident			& El identificador del subprograma. \\ \cline{2-4}
							& \ter{H}		& hrequireFunc	& Requerimos que sea una función. \\ \cline{2-4}
							& \ter{S}		& tipoRetorno	& El tipo de retorno. \\ \hline

	expresion				& \ter{S} 		& nombre			& Nombre de la variable del programa o temporal que contiene el
														  	  valor de evaluar la expresión.  \\ \cline{2-4} 
							& \ter{S}		& tipo			& El tipo de la expresión. \\ \cline{2-4}
							& \ter{S}		& literal		& Marca si el contenido es un literal. \\ \cline{2-4}
							& \ter{S} 		& true			& Lista de referencias a saltos que han de completarse para definir a 
														  	  dónde saltará el programa si la \emph{evaluación de la expresión es cierta}. \\ \cline{2-4} 
							& \ter{S} 		& false			& Lista de referencias a saltos que han de completarse para definir a 
															  dónde saltará el programa si la \emph{evaluación de la expresión es falsa}. \\ \hline	

	disyuncion				& \ter{S} 		& nombre			& Nombre de la variable del programa o temporal que contiene el
														  	  valor de evaluar la expresión.  \\ \cline{2-4} 
							& \ter{S}		& tipo			& El tipo de la expresión. \\ \cline{2-4}
							& \ter{S}		& literal		& Marca si el contenido es un literal. \\ \cline{2-4}
							& \ter{S} 		& true			& Lista de referencias a saltos que han de completarse para definir a 
														  	  dónde saltará el programa si la \emph{evaluación de la expresión es cierta}. \\ \cline{2-4} 
							& \ter{S} 		& false			& Lista de referencias a saltos que han de completarse para definir a 
															  dónde saltará el programa si la \emph{evaluación de la expresión es falsa}. \\ \hline

	conjuncion				& \ter{S} 		& nombre			& Nombre de la variable del programa o temporal que contiene el
														  	  valor de evaluar la expresión.  \\ \cline{2-4} 
							& \ter{S}		& tipo			& El tipo de la expresión. \\ \cline{2-4}
							& \ter{S}		& literal		& Marca si el contenido es un literal. \\ \cline{2-4}
							& \ter{S} 		& true			& Lista de referencias a saltos que han de completarse para definir a 
														  	  dónde saltará el programa si la \emph{evaluación de la expresión es cierta}. \\ \cline{2-4} 
							& \ter{S} 		& false			& Lista de referencias a saltos que han de completarse para definir a 
															  dónde saltará el programa si la \emph{evaluación de la expresión es falsa}. \\ \hline
															  
\end{tabularx}

\vfill

\begin{tabularx}{\textwidth}{| r | c | c | X |} \hline

	\ter{No terminal}	& \ter{Tipo}		& \ter{Nombre}	& \ter{Descripcion} \\ \hline \hline	
	
	%-----
											  
	aritmética				& \ter{S} 		& nombre			& Nombre de la variable del programa o temporal que contiene el
														  	  valor de evaluar la expresión.  \\ \cline{2-4} 
							& \ter{S}		& tipo			& El tipo de la expresión. \\ \cline{2-4}
							& \ter{S}		& literal		& Marca si el contenido es un literal. \\ \cline{2-4}
							& \ter{S} 		& true			& Lista de referencias a saltos que han de completarse para definir a 
														  	  dónde saltará el programa si la \emph{evaluación de la expresión es cierta}. \\ \cline{2-4} 
							& \ter{S} 		& false			& Lista de referencias a saltos que han de completarse para definir a 
															  dónde saltará el programa si la \emph{evaluación de la expresión es falsa}. \\ \hline
															  
	termino					& \ter{S} 		& nombre			& Nombre de la variable del programa o temporal que contiene el
														  	  valor de evaluar la expresión.  \\ \cline{2-4} 
							& \ter{S}		& tipo			& El tipo de la expresión. \\ \cline{2-4}
							& \ter{S}		& literal		& Marca si el contenido es un literal. \\ \cline{2-4}
							& \ter{S} 		& true			& Lista de referencias a saltos que han de completarse para definir a 
														  	  dónde saltará el programa si la \emph{evaluación de la expresión es cierta}. \\ \cline{2-4} 
							& \ter{S} 		& false			& Lista de referencias a saltos que han de completarse para definir a 
															  dónde saltará el programa si la \emph{evaluación de la expresión es falsa}. \\ \hline
															  
	factor					& \ter{S} 		& nombre			& Nombre de la variable del programa o temporal que contiene el
														  	  valor de evaluar la expresión.  \\ \cline{2-4} 
							& \ter{S}		& tipo			& El tipo de la expresión. \\ \cline{2-4}
							& \ter{S}		& literal		& Marca si el contenido es un literal. \\ \cline{2-4}
							& \ter{S} 		& true			& Lista de referencias a saltos que han de completarse para definir a 
														  	  dónde saltará el programa si la \emph{evaluación de la expresión es cierta}. \\ \cline{2-4} 
							& \ter{S} 		& false			& Lista de referencias a saltos que han de completarse para definir a 
															  dónde saltará el programa si la \emph{evaluación de la expresión es falsa}. \\ \hline
															  
	factor'					& \ter{S} 		& nombre			& Nombre de la variable del programa o temporal que contiene el
														  	  valor de evaluar la expresión.  \\ \cline{2-4} 
							& \ter{S}		& tipo			& El tipo de la expresión. \\ \cline{2-4}
							& \ter{S}		& literal		& Marca si el contenido es un literal. \\ \cline{2-4}
							& \ter{S} 		& true			& Lista de referencias a saltos que han de completarse para definir a 
														  	  dónde saltará el programa si la \emph{evaluación de la expresión es cierta}. \\ \cline{2-4} 
							& \ter{S} 		& false			& Lista de referencias a saltos que han de completarse para definir a 
															  dónde saltará el programa si la \emph{evaluación de la expresión es falsa}. \\ \hline
															  
\end{tabularx}

\vfill

\begin{tabularx}{\textwidth}{| r | c | c | X |} \hline

	\ter{No terminal}	& \ter{Tipo}		& \ter{Nombre}	& \ter{Descripcion} \\ \hline \hline	
	
	%-----
	
	disyuncion'				& \ter{H} 		& hnombre		& Nombre del operando izquierdo.  \\ \cline{2-4} 
							& \ter{H}		& htipo			& Tipo del operando izquierdo. \\ \cline{2-4}
							& \ter{H}		& hliteral		& Marca si el operando izquierdo es un literal. \\ \cline{2-4}
							& \ter{H} 		& htrue			& Lista de referencias a saltos true del operando izquierdo. \\ \cline{2-4} 
							& \ter{H} 		& hfalse			& Lista de referencias a saltos false del operando izquierdo. \\ 
							& \ter{S} 		& nombre			& Nombre de la variable del programa o temporal que contiene el
														  	  valor de evaluar la expresión.  \\ \cline{2-4} 
							& \ter{S}		& tipo			& El tipo de la expresión. \\ \cline{2-4}
							& \ter{S}		& literal		& Marca si el contenido es un literal. \\ \cline{2-4}
							& \ter{S} 		& true			& Lista de referencias a saltos que han de completarse para definir a 
														  	  dónde saltará el programa si la \emph{evaluación de la expresión es cierta}. \\ \cline{2-4} 
							& \ter{S} 		& false			& Lista de referencias a saltos que han de completarse para definir a 
															  dónde saltará el programa si la \emph{evaluación de la expresión es falsa}. \\ \hline
															  
	conjuncion'				& \ter{H} 		& hnombre		& Nombre del operando izquierdo.  \\ \cline{2-4} 
							& \ter{H}		& htipo			& Tipo del operando izquierdo. \\ \cline{2-4}
							& \ter{H}		& hliteral		& Marca si el operando izquierdo es un literal. \\ \cline{2-4}
							& \ter{H} 		& htrue			& Lista de referencias a saltos true del operando izquierdo. \\ \cline{2-4} 
							& \ter{H} 		& hfalse			& Lista de referencias a saltos false del operando izquierdo. \\ 
							& \ter{S} 		& nombre			& Nombre de la variable del programa o temporal que contiene el
														  	  valor de evaluar la expresión.  \\ \cline{2-4} 
							& \ter{S}		& tipo			& El tipo de la expresión. \\ \cline{2-4}
							& \ter{S}		& literal		& Marca si el contenido es un literal. \\ \cline{2-4}
							& \ter{S} 		& true			& Lista de referencias a saltos que han de completarse para definir a 
														  	  dónde saltará el programa si la \emph{evaluación de la expresión es cierta}. \\ \cline{2-4} 
							& \ter{S} 		& false			& Lista de referencias a saltos que han de completarse para definir a 
															  dónde saltará el programa si la \emph{evaluación de la expresión es falsa}. \\ \hline
\end{tabularx}

\vfill

\begin{tabularx}{\textwidth}{| r | c | c | X |} \hline

	\ter{No terminal}	& \ter{Tipo}		& \ter{Nombre}	& \ter{Descripcion} \\ \hline \hline	
	
	%-----
											  
	relacional'				& \ter{H} 		& hnombre		& Nombre del operando izquierdo.  \\ \cline{2-4} 
							& \ter{H}		& htipo			& Tipo del operando izquierdo. \\ \cline{2-4}
							& \ter{H}		& hliteral		& Marca si el operando izquierdo es un literal. \\ \cline{2-4}
							& \ter{H} 		& htrue			& Lista de referencias a saltos true del operando izquierdo. \\ \cline{2-4} 
							& \ter{H} 		& hfalse			& Lista de referencias a saltos false del operando izquierdo. \\ 
							& \ter{S} 		& nombre			& Nombre de la variable del programa o temporal que contiene el
														  	  valor de evaluar la expresión.  \\ \cline{2-4} 
							& \ter{S}		& tipo			& El tipo de la expresión. \\ \cline{2-4}
							& \ter{S}		& literal		& Marca si el contenido es un literal. \\ \cline{2-4}
							& \ter{S} 		& true			& Lista de referencias a saltos que han de completarse para definir a 
														  	  dónde saltará el programa si la \emph{evaluación de la expresión es cierta}. \\ \cline{2-4} 
							& \ter{S} 		& false			& Lista de referencias a saltos que han de completarse para definir a 
															  dónde saltará el programa si la \emph{evaluación de la expresión es falsa}. \\ \hline
															  
	aritmetica'				& \ter{H} 		& hnombre		& Nombre del operando izquierdo.  \\ \cline{2-4} 
							& \ter{H}		& htipo			& Tipo del operando izquierdo. \\ \cline{2-4}
							& \ter{H}		& hliteral		& Marca si el operando izquierdo es un literal. \\ \cline{2-4}
							& \ter{H} 		& htrue			& Lista de referencias a saltos true del operando izquierdo. \\ \cline{2-4} 
							& \ter{H} 		& hfalse			& Lista de referencias a saltos false del operando izquierdo. \\ 
							& \ter{S} 		& nombre			& Nombre de la variable del programa o temporal que contiene el
														  	  valor de evaluar la expresión.  \\ \cline{2-4} 
							& \ter{S}		& tipo			& El tipo de la expresión. \\ \cline{2-4}
							& \ter{S}		& literal		& Marca si el contenido es un literal. \\ \cline{2-4}
							& \ter{S} 		& true			& Lista de referencias a saltos que han de completarse para definir a 
														  	  dónde saltará el programa si la \emph{evaluación de la expresión es cierta}. \\ \cline{2-4} 
							& \ter{S} 		& false			& Lista de referencias a saltos que han de completarse para definir a 
															  dónde saltará el programa si la \emph{evaluación de la expresión es falsa}. \\ \hline
\end{tabularx}

\vfill

\begin{tabularx}{\textwidth}{| r | c | c | X |} \hline

	\ter{No terminal}	& \ter{Tipo}		& \ter{Nombre}	& \ter{Descripcion} \\ \hline \hline	
	
	%-----
														  
	termino'					& \ter{H} 		& hnombre		& Nombre del operando izquierdo.  \\ \cline{2-4} 
							& \ter{H}		& htipo			& Tipo del operando izquierdo. \\ \cline{2-4}
							& \ter{H}		& hliteral		& Marca si el operando izquierdo es un literal. \\ \cline{2-4}
							& \ter{H} 		& htrue			& Lista de referencias a saltos true del operando izquierdo. \\ \cline{2-4} 
							& \ter{H} 		& hfalse			& Lista de referencias a saltos false del operando izquierdo. \\ 
							& \ter{S} 		& nombre			& Nombre de la variable del programa o temporal que contiene el
														  	  valor de evaluar la expresión.  \\ \cline{2-4} 
							& \ter{S}		& tipo			& El tipo de la expresión. \\ \cline{2-4}
							& \ter{S}		& literal		& Marca si el contenido es un literal. \\ \cline{2-4}
							& \ter{S} 		& true			& Lista de referencias a saltos que han de completarse para definir a 
														  	  dónde saltará el programa si la \emph{evaluación de la expresión es cierta}. \\ \cline{2-4} 
							& \ter{S} 		& false			& Lista de referencias a saltos que han de completarse para definir a 
															  dónde saltará el programa si la \emph{evaluación de la expresión es falsa}. \\ \hline
															  
	array\_o\_llamada		& \ter{H}		& hindent		& Identificador del array o del subprograma. \\ \cline{2-4}
							& \ter{S}		& nombre			& Identificador en el que se almacena el resultado. \\ \cline{2-4}
							& \ter{S}		& tipo			& Tipo del resultado. \\ \hline
							
	acceso\_a\_array\_opcional	& \ter{H}	& hnombre		& Identificador del array. \\ \cline{2-4}
								& \ter{H}	& htipo			& Tipo del identificador. \\ \cline{2-4}
								& \ter{S}	& nombre			& Identificador en el que se almacena el resultado. \\ \cline{2-4}
								& \ter{S}	& tipo			& Tipo del resultado. \\ \hline
	
	lista\_de\_expr				& \ter{S}	& exprs			& Lista de identificadores. \\ \cline{2-4}
								& \ter{S}	& tipos			& Tipos de las expresiones. \\ \cline{2-4}
								& \ter{S}	& literales		& Lista de marcadores de ser literal. \\ \hline

	resto\_lista\_expr			& \ter{S}	& exprs			& Lista de identificadores. \\ \cline{2-4}
								& \ter{S}	& tipos			& Tipos de las expresiones. \\ \cline{2-4}
								& \ter{S}	& literales		& Lista de marcadores de ser literal. \\ \hline

	opl2						& \ter{S}		& op			& Contiene el nombre del \emph{operador aritmético} (+ o -). \\ \hline

	opl1						& \ter{S}		& op			& Contiene el nombre del \emph{operador aritmético} (* o /). \\ \hline

	oprel 					& \ter{S}		& op			& Contiene el nombre del \emph{operador relacional}. \\ \hline
	
	booleano					& \ter{S}		& value		& Valor literal del booleano. \\ \hline
	
	M						& \ter{S} 		& ref			& Referencia a la instrucción que se introducirá a continuación. \\ \hline
	
\end{tabularx}
