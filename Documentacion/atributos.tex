\chapter{Atributos}

Lista de atributos, detallando para cada uno una descripción y la naturaleza del atributo (léxico, sintetizado o heredado) :

\begin{enumerate}
	\item \ter{L} = léxico 
	\item \ter{S} = sintetizado 
	\item \ter{H} = heredado 
\end{enumerate}

\section*{Listado de atributos}

\begin{tabularx}{\textwidth}{| r | c | c | X |} \hline

	\ter{No terminal}	        & \ter{Tipo}		& \ter{Nombre}	& \ter{Descripcion} \\ \hline \hline
	
	%-----
	
	\ter{Todos los tipos de token} & \ter{L} 	& value			& El valor literal del token encontrado. \\ \hline
		
	programa 			        &&& \\ \hline

    lista\_de\_ident 	        & \ter{S} 		& ids			& Lista que contiene los identificadores encontrados. \\ \hline

    resto\_lista\_ident	        & \ter{S} 		& ids			& Lista que contiene los identificadores encontrados. \\ \hline
	
	declaraciones 		        &&& \\ \hline
	
	tipo 				        & \ter{S} 		& tipo 			& El nombre interno del tipo especificado en la sintaxis. \\ \hline
    
    lista\_de\_enteros          & \ter{S}       & ints          & Lista que contiene los enteros que se van encontrando. \\ \hline

    resto\_lista\_enteros       & \ter{S}       & ints          & Lista que contiene los enteros que se van encontrando. \\ \hline

	decl\_de\_subprogs 	        &&& \\ \hline
	
	\multirow{3}{*}{decl\_de\_procedimiento}
                                & \ter{S}       & args              & Lista que contiene los argumentos del procedimiento. \\ \cline{2-4}
                                & \ter{S}       & classes              & \\ \cline{2-4}     
                                & \ter{S}       & nombre            & Contiene el nombre del procedimiento. \\ \hline

    \multirow{4}{*}{decl\_de\_function}
                                & \ter{S}       & args              & Lista que contiene los argumentos de la función. \\ \cline{2-4}
                                &               & classes           & \\ \cline{2-4}  
                                & \ter{S}       & nombre            & Contiene el nombre de la función. \\ \hline
                                & \ter{L}       & tipoRetorno       & Contiene el tipo de dato de retorno de la función. \\ \hline

	\multirow{3}{*}{cabecera\_procedimiento}
                                &               & args              & \\ \cline{2-4}
                                &               & classes              & \\ \cline{2-4}     
                                &               & nombre             & \\ \hline

	\multirow{4}{*}{cabecera\_funcion}
                                &               & args              & \\ \cline{2-4}
                                &               & classes              & \\ \cline{2-4}  
                                &               & nombre              & \\ \cline{2-4}         
                                &               & tipoRetorno              & \\ \hline
	
	\multirow{2}{*}{argumentos}
                                &               & args              & \\ \cline{2-4}
                                &               & classes              & \\ \hline
	
	\multirow{2}{*}{lista\_de\_param}
                                &               & args              & \\ \cline{2-4}
                                &               & classes              & \\ \hline

	\multirow{2}{*}{resto\_lis\_de\_param}
                                &               & args              & \\ \cline{2-4}
                                &               & classes              & \\ \hline
	
	clase\_param			    & \ter{S} 		& clase			& La clase de parámetro que se ha derivado de la sintaxis (referencia o valor). \\ \hline

	clase\_param’			    & \ter{S} 		& clase			& La clase de parámetro que se ha derivado de la sintaxis (referencia o valor). \\ \hline
	
	lista\_de\_sentencias’      & \ter{H}       & hinloop       & Marcador que indica si estamos dentro de un bucle. \\ \hline
	
\end{tabularx}

\vfill

\clearpage

\begin{tabularx}{\textwidth}{| r | c | c | X |} \hline

	\ter{No terminal}	        & \ter{Tipo}		& \ter{Nombre}	& \ter{Descripcion} \\ \hline \hline	
	
	%-----

	\multirow{2}{*}{lista\_de\_sentencias}  
						        & \ter{H} 		& hinloop 		& Marcador que indica si estamos dentro de un bucle. \\ \cline{2-4} 
						        & \ter{S} 		& salir\_si		& Lista de referencias a las instrucciones "salir\_si" contenidas dentro del ámbito de lista de    sentencias. \\ \hline
						
	sentencia 			        & \ter{S} 		& salir\_si		& Lista de referencias a las instrucciones "salir\_si" contenidas dentro de la sentencia. \\ \hline

    \multirow{2}{*}{id\_o\_array}                
                                & \ter{S}       & nombre        & \cline{2-4}
                                & \ter{H}       & hident        & \\ \hline

    asignacion\_o\_llamada      & \ter{H}       & hident        & \\ \hline

    \multirow{2}{*}{acceso\_a\_array'}           
                                & \ter{S}       & exprs         & \\ \cline{2-4}
                                & \ter{S}       & tipos         & \\ \hline

    \mutirow{4}{*}{acceso\_a\_array}
                                & \ter{S}       & offset        & \\ \cline{2-4}
                                & \ter{H}       & htipo         & \\ \cline{2-4}
                                & \ter{H}       & hident        & \\ \cline{2-4}
                                & \ter{S}       & tipo          & \\ \hline

    \multirow{3}{*}{parametros\_llamadas}        
                                & \ter{S}       & tipoRetorno   & \\ \cline{2-4}
                                & \ter{H}       & hident        & \\ \cline{2-4}
                                & \ter{H}       & hRequireFunc  & \\ \hline

	variable				    & \ter{S} 		& nombre			& El nombre del identificador que define esta variable. \\ \hline

	\multirow{5}{*}{expresion} 
						        & \ter{S} 		& nombre			& Nombre de la variable del programa o temporal que contiene el
														  valor de evaluar la expresión.  \\ \cline{2-4} 
						        & \ter{S} 		& gtrue			& Lista de referencias a saltos que han de completarse para definir a 
														  dónde saltará el programa si la \emph{evaluación de la expresión es cierta}. \\ \cline{2-4} 
						        & \ter{S} 		& gfalse			& Lista de referencias a saltos que han de completarse para definir a 
														dónde saltará el programa si la \emph{evaluación de la expresión es falsa}. \\ \cline{2-4}
                                & \ter{S}       & tipo              & \\ \cline{2-4}
                                & \ter{S}       & literal           & Indica si la expresión a tratar es un literal o no \\ \hline
    \multirow{5}{*}{disyuncion} 
						        & \ter{S} 		& nombre			& Nombre de la variable del programa o temporal que contiene el
														  valor de evaluar la expresión.  \\ \cline{2-4} 
						        & \ter{S} 		& gtrue			& Lista de referencias a saltos que han de completarse para definir a 
														  dónde saltará el programa si la \emph{evaluación de la expresión es cierta}. \\ \cline{2-4} 
						        & \ter{S} 		& gfalse			& Lista de referencias a saltos que han de completarse para definir a 
														dónde saltará el programa si la \emph{evaluación de la expresión es falsa}. \\ \cline{2-4}
                                & \ter{S}       & tipo              & \\ \cline{2-4}
                                & \ter{S}       & literal           & Indica si la disyunción a tratar es un literal o no \\ \hline

    \multirow{5}{*}{conjuncion} 
						        & \ter{S} 		& nombre			& Nombre de la variable del programa o temporal que contiene el
														  valor de evaluar la expresión.  \\ \cline{2-4} 
						        & \ter{S} 		& gtrue			& Lista de referencias a saltos que han de completarse para definir a 
														  dónde saltará el programa si la \emph{evaluación de la expresión es cierta}. \\ \cline{2-4} 
						        & \ter{S} 		& gfalse			& Lista de referencias a saltos que han de completarse para definir a 
														dónde saltará el programa si la \emph{evaluación de la expresión es falsa}. \\ \cline{2-4}
                                & \ter{S}       & tipo              & \\ \cline{2-4}
                                & \ter{S}       & literal           & Indica si la conjunción a tratar es un literal o no \\ \hline
    
    \multirow{5}{*}{relacional} 
						        & \ter{S} 		& nombre			& Nombre de la variable del programa o temporal que contiene el
														  valor de evaluar la expresión.  \\ \cline{2-4} 
						        & \ter{S} 		& gtrue			& Lista de referencias a saltos que han de completarse para definir a 
														  dónde saltará el programa si la \emph{evaluación de la expresión es cierta}. \\ \cline{2-4} 
						        & \ter{S} 		& gfalse			& Lista de referencias a saltos que han de completarse para definir a 
														dónde saltará el programa si la \emph{evaluación de la expresión es falsa}. \\ \cline{2-4}
                                & \ter{S}       & tipo              & \\ \cline{2-4}
                                & \ter{S}       & literal           & Indica si la operación relacional a tratar es un literal o no \\ \hline
    \multirow{5}{*}{aritmetica} 
						        & \ter{S} 		& nombre			& Nombre de la variable del programa o temporal que contiene el
														  valor de evaluar la expresión.  \\ \cline{2-4} 
						        & \ter{S} 		& gtrue			& Lista de referencias a saltos que han de completarse para definir a 
														  dónde saltará el programa si la \emph{evaluación de la expresión es cierta}. \\ \cline{2-4} 
						        & \ter{S} 		& gfalse			& Lista de referencias a saltos que han de completarse para definir a 
														dónde saltará el programa si la \emph{evaluación de la expresión es falsa}. \\ \cline{2-4}
                                & \ter{S}       & tipo              & \\ \cline{2-4}
                                & \ter{S}       & literal           & Indica si la operación aritmética a tratar es un literal o no \\ \hline

    \multirow{5}{*}{termino} 
						        & \ter{S} 		& nombre			& Nombre de la variable del programa o temporal que contiene el
														  valor de evaluar la expresión.  \\ \cline{2-4} 
						        & \ter{S} 		& gtrue			& Lista de referencias a saltos que han de completarse para definir a 
														  dónde saltará el programa si la \emph{evaluación de la expresión es cierta}. \\ \cline{2-4} 
						        & \ter{S} 		& gfalse			& Lista de referencias a saltos que han de completarse para definir a 
														dónde saltará el programa si la \emph{evaluación de la expresión es falsa}. \\ \cline{2-4}
                                & \ter{S}       & tipo              & \\ \cline{2-4}
                                & \ter{S}       & literal           & Indica si el término a tratar es un literal o no \\ \hline

    \multirow{10}{*}{disyuncion'} 
						        & \ter{S} 		& nombre			& Nombre de la variable del programa o temporal que contiene el
														  valor de evaluar la expresión.  \\ \cline{2-4} 
						        & \ter{S} 		& gtrue			& Lista de referencias a saltos que han de completarse para definir a 
														  dónde saltará el programa si la \emph{evaluación de la expresión es cierta}. \\ \cline{2-4} 
						        & \ter{S} 		& gfalse			& Lista de referencias a saltos que han de completarse para definir a 
														dónde saltará el programa si la \emph{evaluación de la expresión es falsa}. \\ \cline{2-4}
                                & \ter{S}       & tipo              & \\ \cline{2-4}
                                & \ter{S}       & literal           & Indica si la disyunción a tratar es un literal o no \\ \cline{2-4}
						        & \ter{H} 		& hnombre			& Nombre de la variable del programa o temporal que contiene el
														  valor de evaluar la expresión.  \\ \cline{2-4} 
						        & \ter{H} 		& hgtrue			& Lista de referencias a saltos que han de completarse para definir a 
														  dónde saltará el programa si la \emph{evaluación de la expresión es cierta}. \\ \cline{2-4} 
						        & \ter{H} 		& hgfalse			& Lista de referencias a saltos que han de completarse para definir a 
														dónde saltará el programa si la \emph{evaluación de la expresión es falsa}. \\ \cline{2-4}
                                & \ter{H}       & htipo              & \\ \cline{2-4}
                                & \ter{H}       & hliteral           & Indica si la disyunción tratada anteriormente es un literal o no \\ \hline

    \multirow{10}{*}{conjuncion'} 
						        & \ter{S} 		& nombre			& Nombre de la variable del programa o temporal que contiene el
														  valor de evaluar la expresión.  \\ \cline{2-4} 
						        & \ter{S} 		& gtrue			& Lista de referencias a saltos que han de completarse para definir a 
														  dónde saltará el programa si la \emph{evaluación de la expresión es cierta}. \\ \cline{2-4} 
						        & \ter{S} 		& gfalse			& Lista de referencias a saltos que han de completarse para definir a 
														dónde saltará el programa si la \emph{evaluación de la expresión es falsa}. \\ \cline{2-4}
                                & \ter{S}       & tipo              & \\ \cline{2-4}
                                & \ter{S}       & literal           & Indica si la conjunción a tratar es un literal o no \\ \cline{2-4}
						        & \ter{H} 		& hnombre			& Nombre de la variable del programa o temporal que contiene el
														  valor de evaluar la expresión.  \\ \cline{2-4} 
						        & \ter{H} 		& hgtrue			& Lista de referencias a saltos que han de completarse para definir a 
														  dónde saltará el programa si la \emph{evaluación de la expresión es cierta}. \\ \cline{2-4} 
						        & \ter{H} 		& hgfalse			& Lista de referencias a saltos que han de completarse para definir a 
														dónde saltará el programa si la \emph{evaluación de la expresión es falsa}. \\ \cline{2-4}
                                & \ter{H}       & htipo              & \\ \cline{2-4}
                                & \ter{H}       & hliteral           & Indica si la conjunción tratada anteriormente es un literal o no \\ \hline
    
    \multirow{10}{*}{relacional'} 
						        & \ter{S} 		& nombre			& Nombre de la variable del programa o temporal que contiene el
														  valor de evaluar la expresión.  \\ \cline{2-4} 
						        & \ter{S} 		& gtrue			& Lista de referencias a saltos que han de completarse para definir a 
														  dónde saltará el programa si la \emph{evaluación de la expresión es cierta}. \\ \cline{2-4} 
						        & \ter{S} 		& gfalse			& Lista de referencias a saltos que han de completarse para definir a 
														dónde saltará el programa si la \emph{evaluación de la expresión es falsa}. \\ \cline{2-4}
                                & \ter{S}       & tipo              & \\ \cline{2-4}
                                & \ter{S}       & literal           & Indica si la operación relacional a tratar es un literal o no \\ \cline{2-4}
						        & \ter{H} 		& hnombre			& Nombre de la variable del programa o temporal que contiene el
														  valor de evaluar la expresión.  \\ \cline{2-4} 
						        & \ter{H} 		& hgtrue			& Lista de referencias a saltos que han de completarse para definir a 
														  dónde saltará el programa si la \emph{evaluación de la expresión es cierta}. \\ \cline{2-4} 
						        & \ter{H} 		& hgfalse			& Lista de referencias a saltos que han de completarse para definir a 
														dónde saltará el programa si la \emph{evaluación de la expresión es falsa}. \\ \cline{2-4}
                                & \ter{H}       & htipo              & \\ \cline{2-4}
                                & \ter{H}       & hliteral           & Indica si la operación relacional tratada anteriormente es un literal o no \\ \hline

    \multirow{10}{*}{aritmetica'} 
						        & \ter{S} 		& nombre			& Nombre de la variable del programa o temporal que contiene el
														  valor de evaluar la expresión.  \\ \cline{2-4} 
						        & \ter{S} 		& gtrue			& Lista de referencias a saltos que han de completarse para definir a 
														  dónde saltará el programa si la \emph{evaluación de la expresión es cierta}. \\ \cline{2-4} 
						        & \ter{S} 		& gfalse			& Lista de referencias a saltos que han de completarse para definir a 
														dónde saltará el programa si la \emph{evaluación de la expresión es falsa}. \\ \cline{2-4}
                                & \ter{S}       & tipo              & \\ \cline{2-4}
                                & \ter{S}       & literal           & Indica si la operación aritmética a tratar es un literal o no \\ \cline{2-4}
						        & \ter{H} 		& hnombre			& Nombre de la variable del programa o temporal que contiene el
														  valor de evaluar la expresión.  \\ \cline{2-4} 
						        & \ter{H} 		& hgtrue			& Lista de referencias a saltos que han de completarse para definir a 
														  dónde saltará el programa si la \emph{evaluación de la expresión es cierta}. \\ \cline{2-4} 
						        & \ter{H} 		& hgfalse			& Lista de referencias a saltos que han de completarse para definir a 
														dónde saltará el programa si la \emph{evaluación de la expresión es falsa}. \\ \cline{2-4}
                                & \ter{H}       & htipo              & \\ \cline{2-4}
                                & \ter{H}       & hliteral           & Indica si la operación aritmética tratada anteriormente es un literal o no \\ \hline
    
    \multirow{10}{*}{termino'} 
						        & \ter{S} 		& nombre			& Nombre de la variable del programa o temporal que contiene el
														  valor de evaluar la expresión.  \\ \cline{2-4} 
						        & \ter{S} 		& gtrue			& Lista de referencias a saltos que han de completarse para definir a 
														  dónde saltará el programa si la \emph{evaluación de la expresión es cierta}. \\ \cline{2-4} 
						        & \ter{S} 		& gfalse			& Lista de referencias a saltos que han de completarse para definir a 
														dónde saltará el programa si la \emph{evaluación de la expresión es falsa}. \\ \cline{2-4}
                                & \ter{S}       & tipo              & \\ \cline{2-4}
                                & \ter{S}       & literal           & Indica si el término a tratar es un literal o no \\ \cline{2-4}
						        & \ter{H} 		& hnombre			& Nombre de la variable del programa o temporal que contiene el
														  valor de evaluar la expresión.  \\ \cline{2-4} 
						        & \ter{H} 		& hgtrue			& Lista de referencias a saltos que han de completarse para definir a 
														  dónde saltará el programa si la \emph{evaluación de la expresión es cierta}. \\ \cline{2-4} 
						        & \ter{H} 		& hgfalse			& Lista de referencias a saltos que han de completarse para definir a 
														dónde saltará el programa si la \emph{evaluación de la expresión es falsa}. \\ \cline{2-4}
                                & \ter{H}       & htipo              & \\ \cline{2-4}
                                & \ter{H}       & hliteral           & Indica si el término tratado anteriormente es un literal o no \\ \hline

    \multirow{5}{*}{negacion} 
						        & \ter{S} 		& nombre			& Nombre de la variable del programa o temporal que contiene el
														  valor de evaluar la expresión.  \\ \cline{2-4} 
						        & \ter{S} 		& gtrue			& Lista de referencias a saltos que han de completarse para definir a 
														  dónde saltará el programa si la \emph{evaluación de la expresión es cierta}. \\ \cline{2-4} 
						        & \ter{S} 		& gfalse			& Lista de referencias a saltos que han de completarse para definir a 
														dónde saltará el programa si la \emph{evaluación de la expresión es falsa}. \\ \cline{2-4}
                                & \ter{S}       & tipo              & \\ \cline{2-4}
                                & \ter{S}       & literal           & Indica si lo que se quiere negar es un literal \\ \hline

    \multirow{5}{*}{factor} 
						        & \ter{S} 		& nombre			& Nombre de la variable del programa o temporal que contiene el
														  valor de evaluar la expresión.  \\ \cline{2-4} 
						        & \ter{S} 		& gtrue			& Lista de referencias a saltos que han de completarse para definir a 
														  dónde saltará el programa si la \emph{evaluación de la expresión es cierta}. \\ \cline{2-4} 
						        & \ter{S} 		& gfalse			& Lista de referencias a saltos que han de completarse para definir a 
														dónde saltará el programa si la \emph{evaluación de la expresión es falsa}. \\ \cline{2-4}
                                & \ter{S}       & tipo              & \\ \cline{2-4}
                                & \ter{S}       & literal           & Indica si el factor a tratar es un literal o no \\ \hline

    \multirow{5}{*}{factor'} 
						        & \ter{S} 		& nombre			& Nombre de la variable del programa o temporal que contiene el
														  valor de evaluar la expresión.  \\ \cline{2-4} 
						        & \ter{S} 		& gtrue			& Lista de referencias a saltos que han de completarse para definir a 
														  dónde saltará el programa si la \emph{evaluación de la expresión es cierta}. \\ \cline{2-4} 
						        & \ter{S} 		& gfalse			& Lista de referencias a saltos que han de completarse para definir a 
														dónde saltará el programa si la \emph{evaluación de la expresión es falsa}. \\ \cline{2-4}
                                & \ter{S}       & tipo              & \\ \cline{2-4}
                                & \ter{S}       & literal           & Indica si el factor a tratar es un literal o no \\ \hline

    array\_o\_llamada           &&& \\ \hline

    acceso\_a\_array\_opcional  &&& \\ \hline

    lista\_de\_expr             &&& \\ \hline

    resto\_lista\_expr          &&& \\ \hline

    opl1					& \ter{S}		& op    			& Contiene el nombre del \emph{operador aritmético} (* o /) \\ \hline

	opl2					& \ter{S}		& op    			& Contiene el nombre del \emph{operador aritmético} (+ o -) \\ \hline
    
    booleano                & \ter{L}       & value             & Contiene el valor de la \emph{constante booleana} (true o false) \\ \hline


	M					    & \ter{S} 		& ref			& Referencia a la instrucción que se introducirá a continuación. \\ \hline

	oprel 				    & \ter{S}		& op			& Contiene el nombre del \emph{operador relacional} ($<$, $>$, $\leq$, $\geq$, $==$ o $/=$) \\ \hline
	
\end{tabularx}
