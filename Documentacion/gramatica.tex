\chapter{Gramática}

% Código para autoformatear código
\lstdefineformat{C}{% 
\{=\newline\string\newline\indent,% 
\}=[;]\newline\noindent\string\newline,% 
\};=\newline\noindent\string\newline,% 
;=[\ ]\string\space}

\newfontfamily{\J}{Andale Mono}

% Establecemos las propiedades de listings
\lstset{
		language={C++}, 
		format={C},
		mathescape,
		texcl,
		breaklines=true,
		basicstyle=\scriptsize,
		escapechar={¡},
}

\small
\begin{tabular}{r c p{.72\textwidth}}
	
	programa 		&$\longrightarrow$	& \ter{programa} \ter{id} \\
					&					& \begin{lstlisting}
										ADD_INST("prog " || ID.value);
										\end{lstlisting} \\
					&					& declaraciones \\
					&					& decl\_de\_subprogs \\
					&					& \ter{comienzo} \\
					&					& lista\_de\_sentencias' \\
					&					& \ter{fin} \ter{;} \\
					&					& \begin{lstlisting}
										ADD_INST("halt");
										\end{lstlisting} \\

	\espacio
	
	lista\_de\_ident	&$\longrightarrow$	& \ter{id} resto\_lista\_ident \\
					&					& \begin{lstlisting}
											lista_de_ident.ids := JOIN(INIT_LIST(id.value), resto_lista_id.ids);
										\end{lstlisting} \\

	\espacio
											
	resto\_lista\_ident 	& $\longrightarrow$ 	& \ter{,} \ter{id} resto\_lista\_ident \\
						&					& \begin{lstlisting}
											resto_lista_indet.ids := 
											JOIN( INIT_LIST(id.value), resto_lista_ident$_{1}$.ids);
											\end{lstlisting} \\
																			
						& | 					& $\xi$ \\
						&					& \begin{lstlisting}
											resto_lista_ident.ids := EMPTY_LIST(); 
											\end{lstlisting} \\

\end{tabular}


\small
\begin{tabular}{r c p{.72\textwidth}}
	declaraciones 	&$\longrightarrow$ 	& \ter{variables} lista\_de\_ident \ter{:} tipo \ter{;} \\
					&					& \begin{lstlisting}
            if ( IS_INTEGER(t.tipo) || IS_REAL(t.tipo) || IS_BOOLEAN(t.tipo) )
            {
                FOREACH ( lista_de_ident.ids AS ident )
                {
                    // Si la política de resolución de colisiones de la tabla de símbolos lo permite 
                    // para este identificador, añadimos la declaración.
                    if ( ST_ADD(ident, tipo.tipo) )
                    {
                        ADD_INST(TYPE_OF(tipo.tipo) || " " || ident );
                    }
                }
            }
            else if ( IS_ARRAY(t.tipo) )
            {
                FOREACH ( lista_de_ident.ids AS ident )
                {
                    // Si la política de resolución de colisiones de la tabla de símbolos lo permite 
                    // para este identificador, añadimos la declaración.
                    if ( ST_ADD(ident, tipo.tipo) )
                    {
                        ADD_INST("array_" || TYPE_OF(ARRAY_CONTENT(tipo.tipo))
                                 || " " || ident || "," || ARRAY_SIZE(tipo.tipo) );
                    }
                }
            }
            
										\end{lstlisting} \\
										
					&					& declaraciones \\
										
					& | 					& $\xi$ \\
					
	\espacio
	
	tipo 			& $\longrightarrow$ 	& \ter{entero} \\
					&					& \begin{lstlisting}
											tipo.tipo := NEW_BASIC_TYPE(INTEGER);
										\end{lstlisting} \\
					
					& | 					& \ter{real} \\
					&					& \begin{lstlisting}
											tipo.tipo := NEW_BASIC_TYPE(REAL);
										\end{lstlisting} \\
					
					& | 					& \ter{booleano} \\
					&					& \begin{lstlisting}
											tipo.tipo := NEW_BASIC_TYPE(BOOLEAN);
										\end{lstlisting} \\

					& | 					& \ter{array} \ter{[} lista\_de\_enteros \ter{]} \ter{de} tipo1 \\
					&					& \begin{lstlisting}
											tipo.tipo := NEW_ARRAY_TYPE(lista_de_enteros.ints, tipo$_{1}$.tipo);
										\end{lstlisting} \\
					
\end{tabular}

\small
\begin{tabular}{r c p{.72\textwidth}}
	
	lista\_de\_enteros	& $\longrightarrow$ 	& entero resto\_lista\_enteros \\
						&					& \begin{lstlisting}
												lista_de_enteros.ints := JOIN( INIT_LIST(entero.value), 
												¡\hspace{18em}¡ resto_lista_enteros.ints);
											\end{lstlisting} \\
					
	\espacio

	resto\_lista\_enteros	& $\longrightarrow$ 	& entero resto\_lista\_enteros \\
							&					& \begin{lstlisting}
												 resto_lista_enteros.ints := JOIN(INIT_LIST(entero.value), 
												 ¡\hspace{19.5em}¡ resto_lista_enteros$_{1}$.ints);
												\end{lstlisting} \\
												 
							& | 					& $\xi$ \\
							&					& \begin{lstlisting}
												resto_de_enteros.ints := EMPTY_LIST();
												\end{lstlisting} \\
					
	\espacio
	
	decl\_de\_subprogs 		& $\longrightarrow$ 	& decl\_de\_procedimiento decl\_de\_subprogs \\
							&					& \begin{lstlisting}
												 ST_ADD ( decl_de_procedimiento.nombre, 
												 ¡\hspace{5em}¡ NEW_PROCEDURE_TYPE(decl_de_procedimiento.args, 
												 ¡\hspace{16.5em}¡ decl_de_procedimiento.classes) );
												\end{lstlisting} \\
												
							& |					& decl\_de\_funcion decl\_de\_subprogs \\
							&					& \begin{lstlisting}
												 ST_ADD ( decl_de_funcion.nombre, 
												 ¡\hspace{5em}¡ NEW_FUNCTION_TYPE( decl_de_funcion.args, 
												 ¡\hspace{16.5em}¡ decl_de_funcion.classes, 
												 ¡\hspace{16.5em}¡ decl_de_funcion.tipoRetorno) );
												\end{lstlisting} \\
							& | 					& $\xi$ \\

\end{tabular}

\small
\begin{tabular}{r c p{.72\textwidth}}

	decl\_de\_procedimiento 	& $\longrightarrow$ 	& \begin{lstlisting}
												 ST_PUSH();
												\end{lstlisting} \\
							&					& cabecera\_procedimiento declaraciones \\
							&					& \ter{comienzo} lista\_de\_sentencias' \ter{fin} \\
							&					& \begin{lstlisting}
												 ADD_INST ( "finproc" );
												\end{lstlisting} \\
							&					& \ter{;} \\
							&					& \begin{lstlisting}
												 ST_POP();
      											 decl_de_procedimiento.args := cabecera_procedimiento.args;
    												 decl_de_procedimiento.classes := cabecera_procedimiento.classes;
      											 decl_de_procedimiento.nombre := cabecera_procedimiento.nombre;
												\end{lstlisting} \\
							
							
	\espacio
	
	decl\_de\_function	 	& $\longrightarrow$ 	& \begin{lstlisting}
												 ST_PUSH();
												\end{lstlisting} \\
							&					& cabecera\_funcion declaraciones \\
							&					& \ter{comienzo} lista\_de\_sentencias' \ter{retornar} expresion \\
							&					&  \begin{lstlisting}
												 ADD_INST ( "ret " || TYPECAST( expresion.nombre, expresion.tipo,
												 ¡\hspace{18em}¡ cabecera_funcion.tipoRetorno, 
												 ¡\hspace{18em}¡ expresion.gtrue, expresion.gfalse ) );
        											 ADD_INST ( "finfunc" );
												\end{lstlisting} \\

							&					& \ter{fin} \\
							&					& \ter{;} \\
							&					& \begin{lstlisting}
												 ST_POP();
												 decl_de_function.args := cabecera_funcion.args;
        											 decl_de_function.classes := cabecera_funcion.classes;
        											 decl_de_function.nombre := cabecera_funcion.nombre;
        											 decl_de_function.tipoRetorno := cabecera_funcion.tipoRetorno;
												\end{lstlisting} \\

\end{tabular}


\small
\begin{tabular}{r c p{.72\textwidth}}

	cabecera	\_procedimiento 	&$\longrightarrow$	& \ter{procedimiento} \ter{id} \\
							& 					& \begin{lstlisting}
												 ADD_INST( "proc " || id.value );
												\end{lstlisting} \\
							&					& argumentos \\
							& 					& \begin{lstlisting}
										         cabecera_procedimiento.nombre := id.value;
        											 cabecera_procedimiento.args := argumentos.args;
    											     cabecera_procedimiento.classes := argumentos.classes;
												\end{lstlisting} \\

	\espacio

	cabecera	\_funcion	 	&$\longrightarrow$	& \ter{funcion} \ter{id} \\
							& 					& \begin{lstlisting}
												 ADD_INST ( "func " || ID.value );
												\end{lstlisting} \\
							&					& argumentos \ter{retorna} tipo \\
							& 					& \begin{lstlisting}
        												cabecera_funcion.tipoRetorno := tipo.tipo;
        												cabecera_funcion.nombre := id.value;
        												cabecera_funcion.args := argumentos.args;
        												cabecera_funcion.classes := argumentos.classes;
												\end{lstlisting} \\
												
	\espacio
		
	argumentos 		&$\longrightarrow$ 	& \ter{(} lista\_de\_param \ter{)} \\
					& 					& \begin{lstlisting}
										 	argumentos.args := lista_de_param.args;
           								 	argumentos.classes := lista_de_param.classes;
										\end{lstlisting} \\

					& | 					& $\xi$ \\
					& 					& \begin{lstlisting}
   										    argumentos.args := EMPTY_LIST ();
    										    argumentos.classes := EMPTY_LIST ();
										\end{lstlisting} \\

							
\end{tabular}


\small
\begin{tabular}{r c p{.72\textwidth}}
	lista\_de\_param &$\longrightarrow$  	& lista\_de\_ident \ter{:} clase\_param tipo \\
					&					& \begin{lstlisting}
       									 lista_de_param.args = EMPTY_LIST ();
   									     lista_de_param.classes = EMPTY_LIST ();
										 FOREACH ( lista_de_ident.ids AS id )
  										 {
											ADD_INST( clase_param.clase || "_" || 
											¡\hspace{3em}¡ TYPE_OF(tipo.tipo) || " " || id );
											ST_ADD ( id, tipo.tipo );
											lista_de_param.args = JOIN(lista_de_param.args, 
											¡\hspace{13em}¡ INIT_LIST(tipo.tipo));
											lista_de_param.classes = JOIN(lista_de_param.classes, 
											¡\hspace{15em}¡ INIT_LIST(clase_param.clase));
										}
										\end{lstlisting} \\
					&					& resto\_lis\_de\_param \\
					&					& \begin{lstlisting}
											lista_de_param.args = JOIN(lista_de_param.args, 
											¡\hspace{16em}¡ resto_lis_de_param.args);
											lista_de_param.classes = JOIN(lista_de_param.classes, 
											¡\hspace{17.5em}¡ resto_lis_de_param.classes);
										\end{lstlisting} \\
	
	resto\_lis\_de\_param 	&$\longrightarrow$ 	& \ter{;} lista\_de\_ident \ter{:} clase\_param tipo \\
							&					& \begin{lstlisting}
           resto_lis_de_param.args := EMPTY_LIST();
            resto_lis_de_param.classes := EMPTY_LIST();
            FOREACH ( lista_de_ident.ids AS id )
            {
                ADD_INST( clase_param.clase || "_" || 
                ¡\hspace{3em}¡ TYPE_OF(tipo.tipo) || " " || id );
                ST_ADD ( id, tipo.tipo );
                resto_lis_de_param.args := JOIN(resto_lis_de_param.args, 
                ¡\hspace{13em}¡ INIT_LIST(tipo.tipo));
                resto_lis_de_param.classes := JOIN(resto_lis_de_param.classes,
                ¡\hspace{15em}¡ INIT_LIST(clase_param.clase));
            }
										\end{lstlisting} \\
							&					& resto\_lis\_de\_param \\
					&						& \begin{lstlisting}
                resto_lis_de_param.args := JOIN(resto_lis_de_param.args, 
                ¡\hspace{18em}¡ resto_lis_de_param$_{1}$.args);
                resto_lis_de_param.classes := JOIN(resto_lis_de_param.classes, 
                ¡\hspace{19em}¡ resto_lis_de_param$_{1}$.classes);

										\end{lstlisting} \\
										
							& |					& $\xi$ \\
							&					& \begin{lstlisting}

                resto_lis_de_param.args := EMPTY_LIST();
                resto_lis_de_param.classes := EMPTY_LIST();

										\end{lstlisting} \\
										
\end{tabular}

\small
\begin{tabular}{r c p{.72\textwidth}}


	clase\_param 	& $\longrightarrow$		& \ter{entrada} clase\_param' \\
					&						& \begin{lstlisting}
												clase_param.clase := clase_param¡'¡.clase;
											\end{lstlisting} \\
					& | 						& \ter{salida} \\
					&						& \begin{lstlisting}
												clase_param.clase := "ref";
											\end{lstlisting} \\
	
	\espacio
	
	clase\_param' 	& $\longrightarrow$		& \ter{salida} \\
					&						& \begin{lstlisting}
												clase_param¡'¡.clase := "ref";
											\end{lstlisting} \\
					& | 						& $\xi$ \\
					&						& \begin{lstlisting}
												clase_param¡'¡.clase := "val";
											\end{lstlisting} \\
					
	lista\_de\_sentencias' 	& $\longrightarrow$ 	& \begin{lstlisting}
												lista_de_sentencias.hinloop := false;
												\end{lstlisting} \\
							& 					& lista\_de\_sentencias \\

	\espacio

	lista\_de\_sentencias 	& $\longrightarrow$ 	&  \begin{lstlisting}
												sentencia.hinloop := lista_de_sentencias.hinloop;
												\end{lstlisting} \\
							& 					& sentencia \\
							&					& \begin{lstlisting}
												lista_de_sentencias$_{1}$.hinloop := lista_de_sentencias.hinloop;
												\end{lstlisting} \\
							&					& lista\_de\_sentencias \\
							&					& \begin{lstlisting}
												lista_de_sentencias.salir_si := JOIN( sentencia.salir_si , 
												¡\hspace{22.5em}¡ lista_de_sentencias$_{1}$.salir_si );
												\end{lstlisting} \\
							& | 					& $\xi$ \\
							&					& \begin{lstlisting}
												lista_de_sentencias.salir_si := EMPTY_LIST();
												\end{lstlisting} \\
\end{tabular}

\small
\begin{tabular}{r c p{.72\textwidth}}

	\espacio

	sentencia 				& $\longrightarrow$ 	& \ter{id} \\
							&					& \begin{lstlisting}
													asignacion_o_llamada.hident := id.value;
												\end{lstlisting} \\
							&					& asignacion\_o\_llamada \ter{;} \\
							&					& \begin{lstlisting}
													sentencia.salir_si := EMPTY_LIST();
												\end{lstlisting} \\
							
							
							& | 					& \ter{si} expresion \\
							&					& \begin{lstlisting}
	TYPECAST( expresion.nombre, expresion.tipo, NEW_BASIC_TYPE(BOOLEXPR), expresion.gtrue, expresion.gfalse);
												\end{lstlisting} \\
							&					& \ter{entonces} M \\
							&					& \begin{lstlisting}
	lista_de_sentencias.hinloop := sentencia.hinloop;
												\end{lstlisting} \\							
							&					& lista\_de\_sentencias \ter{fin} \ter{si} M \ter{;} \\
							&					& \begin{lstlisting}
    COMPLETE( expresion.gtrue , M$_{1}$.ref );
  	COMPLETE( expresion.gfalse, M$_{2}$.ref );
  	sentencia.salir_si := lista_de_sentencias.salir_si;
												\end{lstlisting} \\

							& | 					& \ter{hacer} M \\
							&					& \begin{lstlisting}
	lista_de_sentencias.hinloop := true;
												\end{lstlisting} \\	
							&					& lista\_de\_sentencias \ter{mientras} expresion \\
							&					& \begin{lstlisting}
	TYPECAST( expresion.nombre, expresion.tipo, NEW_BASIC_TYPE(BOOLEXPR), expresion.gtrue, expresion.gfalse);
												\end{lstlisting} \\								
							&					& \ter{fin} \ter{hacer} M \ter{;} \\
							&					& \begin{lstlisting}
            COMPLETE( expresion.gtrue , M$_{1}$.ref );
            COMPLETE( expresion.gfalse, M$_{2}$.ref );
            COMPLETE( lista_de_sentencias.salir_si, M$_{2}$.ref );
            sentencia.salir_si := EMPTY_LIST();
												\end{lstlisting} \\	
						
\end{tabular}

\small
\begin{tabular}{r c p{.72\textwidth}}
														
							& | 					& \ter{salir} \ter{si} \\
							&					& \begin{lstlisting}
	if ( sentencia.hinloop == false )
                ERROR ( "`salir si` outside a loop" );
												\end{lstlisting} \\                
                				&					& expresion M \ter{;} \\
							&					& \begin{lstlisting}
            COMPLETE( expresion.gfalse, M.ref );
            sentencia.salir_si := expresion.gtrue;
												\end{lstlisting} \\
							
							& | 					& \ter{get} \ter{(} \ter{id} \\
							&					& \begin{lstlisting}
			id_o_array.hident := id.value;
												\end{lstlisting} \\
							&					& id\_o\_array \ter{)} \ter{;} \\
							&					& \begin{lstlisting}
            sentencia.salir_si := EMPTY_LIST();
            ADD_INST( "read " || id_o_array.nombre );
												\end{lstlisting} \\
												
							& | 					& \ter{put\_line} \ter{(} expresion \ter{)} \ter{;} \\
							&					& \begin{lstlisting}
            sentencia.salir_si := EMPTY_LIST ();
            ADD_INST ( "write " || expresion.nombre );
            ADD_INST ( "writeln" );
            
												\end{lstlisting} \\

		id\_o\_array			& $\longrightarrow$ 	& \begin{lstlisting}
            acceso_a_array.hident := id_o_array.hident;
            acceso_a_array.htipo := NEW_BASIC_TYPE(UNKNOWN);
												\end{lstlisting} \\
							&					& acceso\_a\_array \\
							&					& \begin{lstlisting}
           id_o_array.nombre = id_o_array.hident || "[" || acceso_a_array.offset || "]";
												\end{lstlisting} \\
												
							& | 					& $\xi$ \\
							&					& \begin{lstlisting}
			if ( ST_EXISTS(id_o_array.hident) == false )
			{
            ERROR("Unknown identifier \\"" || id_o_array.hident || "\\"");
        		}
			id_o_array.nombre = id_o_array.hident;
												\end{lstlisting} \\
												
\end{tabular}

\small
\begin{tabular}{r c p{.72\textwidth}}
		
		asignacion\_o\_llamada	& $\longrightarrow$ 	& \ter{=} expresion \\
								&					& \begin{lstlisting}
            VAR tipo;
            if ( ST_EXISTS (asignacion_o_llamada.hident))
            {
            	tipo := ST_GET_TYPE(asignacion_o_llamada.hident);
            }
            else
            {
            	tipo := NEW_BASIC_TYPE(UNKNOWN);
            	ERROR ( "Identifier " || asignacion_o_llamada.hident || 
			¡\hspace{2em}¡ " not found" );
            }
            ADD_INST( asignacion_o_llamada.hident || " := " || 
            ¡\hspace{6em}¡TYPECAST( expresion.nombre, expresion.tipo, tipo, 
            ¡\hspace{12em}¡expresion.gtrue, expresion.gfalse ) );
 
												\end{lstlisting} \\
												
							& | 					& \begin{lstlisting}
            acceso_a_array.hident := asignacion_o_llamada.hident;
            acceso_a_array.htipo := NEW_BASIC_TYPE(UNKNOWN);
												\end{lstlisting} \\
							& 					& acceso\_a\_array \ter{=} expresion \\
							&  					& \begin{lstlisting}
            ADD_INST ( asignacion_o_llamada.hident || "[" || 
            ¡\hspace{7em}¡acceso_a_array.offset || "]" || " := " || 
            ¡\hspace{7em}¡TYPECAST(expresion.nombre, expresion.tipo, 
            ¡\hspace{12.5em}¡acceso_a_array.tipo) );
												\end{lstlisting} \\
												
							& | 					& \begin{lstlisting}
            parametros_llamadas.hident := acceso_a_array.hident;
            parametros_llamadas.hrequireFunc := false;
												\end{lstlisting} \\
							&					& parametros\_llamadas \\
							
							& 					& \begin{lstlisting}
            ADD_INST( "call " || asignacion_o_llamada.hident );
												\end{lstlisting} \\

	
			acceso\_a\_array'	& $\longrightarrow$ 	& \ter{[} lista\_de\_expr \ter{]} acceso\_a\_array' \\
								&					& \begin{lstlisting}
			acceso_a_array¡'¡.exprs := JOIN(lista_de_expr.exprs, 
			¡\hspace{18em}¡acceso_a_array¡'¡.exprs);
            acceso_a_array¡'¡.tipos := JOIN(lista_de_expr.tipos, 
            ¡\hspace{18em}¡acceso_a_array¡'¡.tipos);
            											\end{lstlisting} \\

								& | 					& $\xi$ \\
								&					& \begin{lstlisting}											
			acceso_a_array¡'¡.exprs := EMPTY_LIST();
        		acceso_a_array¡'¡.tipos := EMPTY_LIST();
                    									\end{lstlisting} \\
																
\end{tabular}

\small
\begin{tabular}{r c p{.72\textwidth}}
										
			acceso\_a\_array		& $\longrightarrow$ 	& \ter{[} lista\_de\_expr \ter{]} acceso\_a\_array' \\
								&					& \begin{lstlisting}
		acceso_a_array.offset := NEW_IDENT ();
        ADD_INST ( acceso_a_array.offset || " := 0" );

        VAR exprs := JOIN(lista_de_expr.exprs, acceso_a_array¡'¡.exprs);
        VAR tipos := JOIN(lista_de_expr.tipos, acceso_a_array¡'¡.tipos);
        
        // No buscamos en la tabla de símbolos si ya nos han provisto de información del tipo
        // a través del atributo heredado, como en el caso de las variables temporales.
        if ( IS_UNKNOWN(acceso_a_array.htipo) == true && ST_EXISTS(acceso_a_array.hident) == false )
        {
            ERROR( "Identifier ¡'¡" || acceso_a_array.hident || "¡'¡ not found." );
            acceso_a_array.tipo := NEW_BASIC_TYPE ( UNKNOWN );
        }
        else
        {
            VAR tipo;
            if ( IS_UNKNOWN(acceso_a_array.htipo) == true )
                tipo := ST_GET_TYPE ( acceso_a_array.hident );
            else
                tipo := acceso_a_array.htipo;
            if ( IS_ARRAY(tipo) == false )
            {
                ERROR( "Identifier ¡'¡" || acceso_a_array.hident || "¡'¡ is not of type ¡'¡array¡'¡.");
                acceso_a_array.tipo := NEW_BASIC_TYPE ( UNKNOWN );
            }
            else
            {
                acceso_a_array.tipo := ARRAY_CONTENT(tipo);
                if ( LIST_SIZE(exprs) > ARRAY_DEPTH(tipo) )
                    ERROR ( "Too many subscripts for array ¡'¡" || acceso_a_array.hident || "¡'¡" );
                else if ( LIST_SIZE(exprs) < ARRAY_DEPTH(tipo) )
                    ERROR ( "Too few subscripts for array ¡'¡" || acceso_a_array.hident || "¡'¡" );
                else
                {
                    VAR i := 0;
                    VAR curDimension := ARRAY_DEPTH(tipo) - 1;
                    FOREACH ( exprs AS nombre )
                    {
                        ADD_INST ( acceso_a_array.offset || " := " || acceso_a_array.offset || " * " || ARRAY_DIMENSION(tipo, curDimension) );
                        ADD_INST ( acceso_a_array.offset || " := " || acceso_a_array.offset || " + " ||
                        		   TYPECAST(nombre, LIST_ITEM(tipos,i), NEW_BASIC_TYPE(INTEGER)) );
                        curDimension := curDimension - 1;
                        i := i + 1;
                    }
                }
            }
        }
 
												\end{lstlisting} \\


\end{tabular}

\small
\begin{tabular}{r c p{.72\textwidth}}
			parametros\_llamadas	& $\longrightarrow$ 	& \ter{(} lista\_de\_expr \ter{)} \\
\end{tabular}

\begin{lstlisting}
        if ( ST_EXISTS ( parametros_llamadas.hident ) == false ) {
            ERROR ( "Unknown identifier ¡'¡" || parametros_llamadas.hident || "¡'¡" );
            parametros_llamadas.tipoRetorno := NEW_BASIC_TYPE ( UNKNOWN );
        } else {
            VAR tipo := ST_GET_TYPE ( parametros_llamadas.hident );
            if ( ( IS_FUNCTION ( tipo ) || IS_PROCEDURE ( tipo ) ) == false )
            {
                ERROR ( "Attempting to call ¡'¡" || parametros_llamadas.hident || "¡'¡, which is not a function or procedure." );
                parametros_llamadas.tipoRetorno := NEW_BASIC_TYPE ( UNKNOWN );
            }
            else if ( parametros_llamadas.hrequireFunc == true && IS_PROCEDURE ( tipo ) )
            {
                ERROR ( "You cannot call ¡'¡" || parametros_llamadas.hident || "¡'¡ from here. It must be a function." );
                parametros_llamadas.tipoRetorno := NEW_BASIC_TYPE ( UNKNOWN );
            }
            else
            {
                if ( IS_PROCEDURE ( tipo ) )
                {
                    parametros_llamadas.tipoRetorno := NEW_BASIC_TYPE ( UNKNOWN );
                    if ( parametros_llamadas.hrequireFunc == true )
                        ERROR ( "You cannot call ¡'¡" || parametros_llamadas.hident || "¡'¡ from here. It must be a function." );
                }
                else
                    parametros_llamadas.tipoRetorno := FUNCTION_RETURN( tipo );

                if ( LIST_SIZE(ls.exprs) != SUBPROG_NUM_PARAMS(tipo) )
                    ERROR ( "Wrong number of arguments for ¡'¡" || parametros_llamadas.hident || "¡'¡" );
                else
                {
                    VAR i := 0;
                    FOREACH ( lista_de_expr.exprs AS expr )
                    {
                        VAR paramTipo := LIST_ITEM(lista_de_expr.tipos, i);
                        VAR reqTipo := SUBPROG_PARAM(tipo, i);
                        if ( SUBPROG_PARAM_CLASS(tipo, i) == "ref" && LIST_ITEM(lista_de_expr.literales, i) == true )
                            ERROR ( "You cannot pass literal values as out parameters." );
                        else if ( SUBPROG_PARAM_CLASS(tipo, i) == "ref" && paramTipo != reqTipo )
                            ERROR ( "Cannot make type conversion for parameter #" || i || " of ¡'¡" || parametros_llamadas.hident || "¡'¡ (Out parameter types must match)" );
                        else
                            ADD_INST ( "param_" || SUBPROG_PARAM_CLASS(tipo, i) || " " || TYPECAST(expr, paramTipo, reqTipo) );
                        i := i + 1;
                    }
                }
            }
        }
\end{lstlisting}

\small
\begin{tabular}{r c p{.72\textwidth}}

			expresion			& $\longrightarrow$ 	& disyuncion \\
								&					& \begin{lstlisting}
        expresion.nombre := disyuncion.nombre;
        expresion.tipo := disyuncion.tipo;
        expresion.gfalse := disyuncion.gfalse;
        expresion.gtrue := disyuncion.gtrue;
        expresion.literal := disyuncion.literal;
                    									\end{lstlisting} \\

			disyuncion			& $\longrightarrow$ 	& conjuncion \\
								&					& \begin{lstlisting}
		disyuncion¡'¡.hnombre := conjuncion.nombre;
        disyuncion¡'¡.htipo := conjuncion.tipo;
        disyuncion¡'¡.hgfalse := conjuncion.gfalse;
        disyuncion¡'¡.hgtrue := conjuncion.gtrue;
        disyuncion¡'¡.hliteral := conjuncion.literal;
                    									\end{lstlisting} \\
								&					& disyuncion' \\
								&					& \begin{lstlisting}
		disyuncion.nombre := disyuncion¡'¡.nombre;
        disyuncion.tipo := disyuncion¡'¡.tipo;
        disyuncion.gfalse := disyuncion¡'¡.gfalse;
        disyuncion.gtrue := disyuncion¡'¡.gtrue;
        disyuncion.literal := disyuncion¡'¡.literal;
                    									\end{lstlisting} \\
		
			conjuncion			& $\longrightarrow$ 	& relacional \\
								&					& \begin{lstlisting}
		conjuncion¡'¡.hnombre := relacional.nombre;
        conjuncion¡'¡.htipo := relacional.tipo;
        conjuncion¡'¡.hgfalse := relacional.gfalse;
        conjuncion¡'¡.hgtrue := relacional.gtrue;
        conjuncion¡'¡.hliteral := relacional.literal;
                    									\end{lstlisting} \\
								&					& conjuncion' \\
								&					& \begin{lstlisting}
		conjuncion.nombre := conjuncion¡'¡.nombre;
        conjuncion.tipo := conjuncion¡'¡.tipo;
        conjuncion.gfalse := conjuncion¡'¡.gfalse;
        conjuncion.gtrue := conjuncion¡'¡.gtrue;
        conjuncion.literal := conjuncion¡'¡.literal;
                    									\end{lstlisting} \\
\end{tabular}


\small
\begin{tabular}{r c p{.72\textwidth}}

		relacional			& $\longrightarrow$ 	& aritmetica \\
								&					& \begin{lstlisting}
		relacional¡'¡.hnombre := aritmetica.nombre;
        relacional¡'¡.htipo := aritmetica.tipo;
        relacional¡'¡.hgfalse := aritmetica.gfalse;
        relacional¡'¡.hgtrue := aritmetica.gtrue;
        relacional¡'¡.hliteral := aritmetica.literal;
                    									\end{lstlisting} \\
								&					& relacional' \\
								&					& \begin{lstlisting}
		relacional.nombre := relacional¡'¡.nombre;
        relacional.tipo := relacional¡'¡.tipo;
        relacional.gfalse := relacional¡'¡.gfalse;
        relacional.gtrue := relacional¡'¡.gtrue;
        relacional.literal := relacional¡'¡.literal;
                    									\end{lstlisting} \\
									
									
		aritmetica			& $\longrightarrow$ 	& termino \\
								&					& \begin{lstlisting}
		aritmetica¡'¡.hnombre := termino.nombre;
        aritmetica¡'¡.htipo := termino.tipo;
        aritmetica¡'¡.hgfalse := termino.gfalse;
        aritmetica¡'¡.hgtrue := termino.gtrue;
        aritmetica¡'¡.hliteral := termino.literal;
                    									\end{lstlisting} \\
								&					& aritmetica' \\
								&					& \begin{lstlisting}
		aritmetica.nombre := aritmetica¡'¡.nombre;
        aritmetica.tipo := aritmetica¡'¡.tipo;
        aritmetica.gfalse := aritmetica¡'¡.gfalse;
        aritmetica.gtrue := aritmetica¡'¡.gtrue;
        aritmetica.literal := aritmetica¡'¡.literal;
                    									\end{lstlisting} \\
									
		
			termino			& $\longrightarrow$ 	& negacion \\
							&					& \begin{lstlisting}
		termino¡'¡.hnombre := negacion.nombre;
        termino¡'¡.htipo := negacion.tipo;
        termino¡'¡.hgfalse := negacion.gfalse;
        termino¡'¡.hgtrue := negacion.gtrue;
        termino¡'¡.hliteral := negacion.literal;
                    									\end{lstlisting} \\
								&					& termino' \\
								&					& \begin{lstlisting}
		termino.nombre := termino¡'¡.nombre;
        termino.tipo := termino¡'¡.tipo;
        termino.gfalse := termino¡'¡.gfalse;
        termino.gtrue := termino¡'¡.gtrue;
        termino.literal := termino¡'¡.literal;
                    									\end{lstlisting} \\
\end{tabular}

\small
\begin{tabular}{r c p{.72\textwidth}}						

			disyuncion'			& $\longrightarrow$ 	& \ter{or} \\
											&					& \begin{lstlisting}
		TYPECAST( disyuncion¡'¡.hnombre, disyuncion¡'¡.htipo, NEW_BASIC_TYPE(BOOLEXPR), disyuncion¡'¡.hgtrue , disyuncion¡'¡.hgfalse );
                    									\end{lstlisting} \\
								&					& M conjuncion \\
								&					& \begin{lstlisting}
        	TYPECAST( conjuncion.nombre, conjuncion.tipo, NEW_BASIC_TYPE(BOOLEXPR), conjuncion.gtrue , conjuncion.gfalse );
        COMPLETE( disyuncion¡'¡.hgfalse, M.ref );
        disyuncion¡'¡$_{1}$.hgtrue := JOIN( disyuncion¡'¡.hgtrue, conjuncion.gtrue );
        disyuncion¡'¡$_{1}$.hgfalse := conjuncion.gfalse;
        disyuncion¡'¡$_{1}$.hnombre := conjuncion.nombre;
        disyuncion¡'¡$_{1}$.htipo := NEW_BASIC_TYPE(BOOLEXPR);
                    									\end{lstlisting} \\
								&					& disyuncion' \\
								&					& \begin{lstlisting}
        disyuncion¡'¡.nombre := disyuncion¡'¡$_{1}$.nombre;
        disyuncion¡'¡.tipo := disyuncion¡'¡$_{1}$.tipo;
        disyuncion¡'¡.gfalse := disyuncion¡'¡$_{1}$.gfalse;
        disyuncion¡'¡.gtrue := disyuncion¡'¡$_{1}$.gtrue;
        disyuncion¡'¡.literal := false;
                    									\end{lstlisting} \\
									
								& | 					& $\xi$ \\
								&					& \begin{lstlisting}
        disyuncion¡'¡.nombre := disyuncion¡'¡.hnombre;
        disyuncion¡'¡.tipo := disyuncion¡'¡.htipo;
        disyuncion¡'¡.gfalse := disyuncion¡'¡.hgfalse;
        disyuncion¡'¡.gtrue := disyuncion¡'¡.hgtrue;
        disyuncion¡'¡.literal := disyuncion¡'¡.hliteral;
                    									\end{lstlisting} \\

\end{tabular}

\small
\begin{tabular}{r c p{.72\textwidth}}

		
			conjuncion'			& $\longrightarrow$ 	& \ter{and} \\
								&					& \begin{lstlisting}
		TYPECAST( conjuncion¡'¡.hnombre, conjuncion¡'¡.htipo, NEW_BASIC_TYPE(BOOLEXPR), conjuncion¡'¡.hgtrue , conjuncion¡'¡.hgfalse );
                    									\end{lstlisting} \\
								&					& M relacional \\
								&					& \begin{lstlisting}
            TYPECAST( relacional.nombre, relacional.tipo, NEW_BASIC_TYPE(BOOLEXPR), relacional.gtrue , relacional.gfalse );
            COMPLETE( conjuncion¡'¡.hgtrue, M.ref );
            conjuncion¡'¡$_{1}$.hgfalse := JOIN( conjuncion¡'¡.hgfalse, relacional.gfalse );
            conjuncion¡'¡$_{1}$.hgtrue := relacional.gtrue;
            conjuncion¡'¡$_{1}$.hnombre := relacional.nombre;
            conjuncion¡'¡$_{1}$.htipo := NEW_BASIC_TYPE(BOOLEXPR);
                    									\end{lstlisting} \\
								&					& conjucion' \\
								&					& \begin{lstlisting}
        conjucion¡'¡.nombre := conjucion¡'¡$_{1}$.nombre;
        conjucion¡'¡.tipo := conjucion¡'¡$_{1}$.tipo;
        conjucion¡'¡.gfalse := conjucion¡'¡$_{1}$.gfalse;
        conjucion¡'¡.gtrue := conjucion¡'¡$_{1}$.gtrue;
        conjucion¡'¡.literal := false;
                    									\end{lstlisting} \\
									
								& | 					& $\xi$ \\
								&					& \begin{lstlisting}
        conjucion¡'¡.nombre := conjucion¡'¡.hnombre;
        conjucion¡'¡.tipo := conjucion¡'¡.htipo;
        conjucion¡'¡.gfalse := conjucion¡'¡.hgfalse;
        conjucion¡'¡.gtrue := conjucion¡'¡.hgtrue;
        conjucion¡'¡.literal := conjucion¡'¡.hliteral;
                    									\end{lstlisting} \\

\end{tabular}


\small
\begin{tabular}{r c p{.72\textwidth}}

			relacional'			& $\longrightarrow$ 	& oprel aritmetica \\
								&					& \begin{lstlisting}
            if ( ! IS_NUMERIC(relacional¡'¡.htipo) ) {
            	ERROR("Relational operand ¡'¡" || relacional¡'¡.hnombre || "¡'¡ is not scalar");
            }
            else if ( ! IS_NUMERIC(aritmetica.tipo) )
            {
            	ERROR("Relational operand ¡'¡" || aritmetica.nombre || "¡'¡ is not scalar");
            }
            else
            {

                if ( IS_REAL(relacional¡'¡.htipo) || IS_REAL(aritmetica.tipo) )
                {
                    TYPECAST( relacional¡'¡.hnombre, relacional¡'¡.htipo, NEW_BASIC_TYPE(REAL) );
                    TYPECAST( aritmetica.nombre, aritmetica.tipo, NEW_BASIC_TYPE(REAL) );
                }

                relacional ¡'¡$_{1}$.hgtrue := INIT_LIST(GET_REF());
                ADD_INST( "if " || relacional¡'¡.hnombre || " " || oprel.op || " " || aritmetica.nombre || " goto ");
                relacional ¡'¡$_{1}$.hgfalse := INIT_LIST(GET_REF());
                ADD_INST( "goto ");

                relacional ¡'¡$_{1}$.htipo = NEW_BASIC_TYPE(BOOLEXPR);
             }
                    									\end{lstlisting} \\
								&					& M relacional' \\
								&					& \begin{lstlisting}
            relacional¡'¡.nombre = relacional¡'¡$_{1}$.nombre;
            relacional¡'¡.tipo = relacional¡'¡$_{1}$.tipo;
            relacional¡'¡.gfalse = relacional¡'¡$_{1}$.gfalse;
            relacional¡'¡.gtrue = relacional¡'¡$_{1}$.gtrue;
            relacional¡'¡.literal = false;
                    									\end{lstlisting} \\
									
								& | 					& $\xi$ \\
								&					& \begin{lstlisting}
        relacional¡'¡.nombre := relacional¡'¡.hnombre;
        relacional¡'¡.tipo := relacional¡'¡.htipo;
        relacional¡'¡.gfalse := relacional¡'¡.hgfalse;
        relacional¡'¡.gtrue := relacional¡'¡.hgtrue;
        relacional¡'¡.literal := relacional¡'¡.hliteral;
                    									\end{lstlisting} \\

\end{tabular}


\small
\begin{tabular}{r c p{.72\textwidth}}

			aritmetica'			& $\longrightarrow$ 	& opl2 termino \\
								&					& \begin{lstlisting}
            if ( IS_NUMERIC(aritmetica¡'¡.htipo) && IS_NUMERIC(termino.tipo) ) {

				// Si uno de los dos es real, se realiza aritmética con números reales.
                if ( IS_REAL(aritmetica¡'¡.htipo) || IS_REAL(termino.tipo) )
                {
                    TYPECAST( aritmetica¡'¡.hnombre, aritmetica¡'¡.htipo, NEW_BASIC_TYPE(REAL));
                    TYPECAST( termino.nombre, termino.tipo, NEW_BASIC_TYPE(REAL));
                    arimetica ¡'¡$_{1}$.htipo := NEW_BASIC_TYPE(REAL);
                }
                else
                {
                    arimetica¡'¡$_{1}$.htipo := NEW_BASIC_TYPE(INTEGER);
                }

                arimetica ¡'¡$_{1}$.hnombre := NEW_IDENT();
                ADD_INST( arimetica¡'¡$_{1}$.hnombre || " := " || arimetica¡'¡.hnombre || " " || opl2.op || " " || termino.nombre );
                
            } else {
                ERROR("Type mismatch error");
                arimetica¡'¡$_{1}$.htipo := NEW_BASIC_TYPE(UNKNOWN);
            }
                    									\end{lstlisting} \\
								&					& aritmetica' \\
								&					& \begin{lstlisting}
            aritmetica¡'¡.nombre = arimetica¡'¡$_{1}$.nombre;
            aritmetica¡'¡.tipo = arimetica¡'¡$_{1}$.tipo;
            aritmetica¡'¡.gfalse = arimetica¡'¡$_{1}$.gfalse;
            aritmetica¡'¡.gtrue = arimetica¡'¡$_{1}$.gtrue;
            aritmetica¡'¡.literal = false;
                    									\end{lstlisting} \\
									
								& | 					& $\xi$ \\
								&					& \begin{lstlisting}
        aritmetica¡'¡.nombre := aritmetica¡'¡.hnombre;
        aritmetica¡'¡.tipo := aritmetica¡'¡.htipo;
        aritmetica¡'¡.gfalse := aritmetica¡'¡.hgfalse;
        aritmetica¡'¡.gtrue := aritmetica¡'¡.hgtrue;
        aritmetica¡'¡.literal := aritmetica¡'¡.hliteral;
                    									\end{lstlisting} \\

\end{tabular}



\small
\begin{tabular}{r c p{.72\textwidth}}

			termino'			& $\longrightarrow$ 	& opl1 negacion \\
							&					& \begin{lstlisting}
			// Si uno de los dos es real, se realiza aritmética con números reales.
            if ( IS_NUMERIC(termino¡'¡.htipo) && IS_NUMERIC(negacion.tipo) ) {

				// Si uno de los dos es real, se realiza aritmética con números reales.
                if ( IS_REAL(termino¡'¡.htipo) || IS_REAL(negacion.tipo) )
                {
                    TYPECAST( termino¡'¡.hnombre, termino¡'¡.htipo, NEW_BASIC_TYPE(REAL) );
                    TYPECAST( negacion.nombre, negacion.tipo, NEW_BASIC_TYPE(REAL) );
                    termino ¡'¡$_{1}$.htipo := NEW_BASIC_TYPE(REAL);
                }
                else
                {
                    termino ¡'¡$_{1}$.htipo := NEW_BASIC_TYPE(INTEGER);
                }

                termino¡'¡$_{1}$.hnombre := NEW_IDENT();
                ADD_INST( termino¡'¡$_{1}$.hnombre || " := " || termino¡'¡.hnombre || " " || opl1.op || " " || negacion.nombre );
                
            } else {
                ERROR("Type mismatch error");
                termino¡'¡$_{1}$.htipo := NEW_BASIC_TYPE(UNKNOWN);
            }
                    									\end{lstlisting} \\
								&					& termino' \\
								&					& \begin{lstlisting}
            termino¡'¡.nombre = termino¡'¡$_{1}$.nombre;
            termino¡'¡.tipo = termino¡'¡$_{1}$.tipo;
            termino¡'¡.gfalse = termino¡'¡$_{1}$.gfalse;
            termino¡'¡.gtrue = termino¡'¡$_{1}$.gtrue;
            termino¡'¡.literal = false;
                    									\end{lstlisting} \\
									
								& | 					& $\xi$ \\
								&					& \begin{lstlisting}
        termino¡'¡.nombre := termino¡'¡.hnombre;
        termino¡'¡.tipo := termino¡'¡.htipo;
        termino¡'¡.gfalse := termino¡'¡.hgfalse;
        termino¡'¡.gtrue := termino¡'¡.hgtrue;
        termino¡'¡.literal := termino¡'¡.hliteral;
                    									\end{lstlisting} \\

\end{tabular}


\small
\begin{tabular}{r c p{.72\textwidth}}

		\espacio
		
			negacion			& $\longrightarrow$ 	& not factor \\
							&					& \begin{lstlisting}
            if ( IS_BOOLEXPR(factor.tipo) )
            {
                negacion.gtrue := factor.gfalse;
                negacion.gfalse := factor.gtrue;
                negacion.tipo := factor.tipo;
            }
            else
            {
                // Le pasamos el gtrue y el gfalse en orden inverso al TYPECAST
                // para que haga de forma transparente la negación.
                negacion.gtrue := EMPTY_LIST ();
                negacion.gfalse := EMPTY_LIST ();
                TYPECAST(factor.nombre, factor.tipo, NEW_BASIC_TYPE(BOOLEXPR), negacion.gfalse, negacion.gtrue);
                negacion.tipo := NEW_BASIC_TYPE(BOOLEXPR);
            }
            negacion.literal = false;
                    									\end{lstlisting} \\
							& |					& factor \\ 
							&					& \begin{lstlisting}
           negacion.nombre := factor.nombre;
           negacion.tipo := factor.tipo;
           negacion.gfalse := factor.gfalse;
           negacion.gtrue := factor.gtrue;
           negacion.literal := factor.literal;
                    									\end{lstlisting} \\

\end{tabular}


\small
\begin{tabular}{r c p{.72\textwidth}}

			factor			& $\longrightarrow$ 	& \ter{-} factor' \\
							&					& \begin{lstlisting}
            if ( IS_REAL(factor¡'¡.tipo) || IS_INTEGER(factor¡'¡.tipo) )
            {
                factor.nombre = NEW_IDENT();
                ADD_INST( factor.nombre || " := 0 - " || factor¡'¡.nombre );
                factor.tipo = factor¡'¡.tipo;
            }
            else
            {
                ERROR("Type mismatch error");
            }
            factor.literal = factor¡'¡.literal;
                    									\end{lstlisting} \\
							&					& factor' \\
							&					& \begin{lstlisting}
           factor.nombre = factor¡'¡.nombre;
           factor.tipo = factor¡'¡.tipo;
           factor.gfalse = factor¡'¡.gfalse;
           factor.gtrue = factor¡'¡.gtrue;
           factor.literal = factor¡'¡.literal;
                    									\end{lstlisting} \\
\end{tabular}


\small
\begin{tabular}{r c p{.72\textwidth}}

		\espacio
		
			factor'			& $\longrightarrow$ 	& \ter{id} \\
							&					& \begin{lstlisting}
				array_o_llamada.hident := IDENTIFIER.value;
                    									\end{lstlisting} \\
							& 					& array\_o\_llamada \\ 
							&					& \begin{lstlisting}
            factor ¡'¡.nombre := array_o_llamada.nombre;
            factor ¡'¡.tipo := array_o_llamada.tipo;
            factor ¡'¡.literal := false;
            factor ¡'¡.gtrue := EMPTY_LIST();
            factor ¡'¡.gfalse := EMPTY_LIST();
                    								\end{lstlisting} \\
								
							& |				 	& \ter{integer} \\
							&					& \begin{lstlisting}
            factor ¡'¡.tipo := NEW_BASIC_TYPE(INTEGER);
            factor ¡'¡.nombre := INTEGER.value;
            factor ¡'¡.literal := true;
            factor ¡'¡.gtrue := EMPTY_LIST();
            factor ¡'¡.gfalse := EMPTY_LIST();
                    									\end{lstlisting} \\
									
							& |				 	& \ter{real} \\
							&					& \begin{lstlisting}
            factor ¡'¡.tipo := NEW_BASIC_TYPE(REAL);
            factor ¡'¡.nombre := REAL.value;
            factor ¡'¡.literal := true;
            factor ¡'¡.gtrue := EMPTY_LIST();
            factor ¡'¡.gfalse := EMPTY_LIST();
                    									\end{lstlisting} \\
									
							& |				 	& booleano \\
							&					& \begin{lstlisting}
            factor ¡'¡.tipo := NEW_BASIC_TYPE(BOOLEAN);
            factor ¡'¡.nombre := booleano.value;
            factor ¡'¡.literal := true;
            factor ¡'¡.gtrue := EMPTY_LIST();
            factor ¡'¡.gfalse := EMPTY_LIST();
                    									\end{lstlisting} \\
									
									
							& |				 	& \ter{(} expresion \ter{)} \\
							&					& \begin{lstlisting}
            factor ¡'¡.tipo := expresion.tipo;
            factor ¡'¡.nombre := expresion.nombre;
            factor ¡'¡.gfalse := expresion.gfalse;
            factor ¡'¡.gtrue := expresion.gtrue;
            factor ¡'¡.literal := expresion.literal;
                    									\end{lstlisting} \\
\end{tabular}


\small
\begin{tabular}{r c p{.72\textwidth}}

		\espacio
		
			array\_o\_llamada	& $\longrightarrow$ 	& \begin{lstlisting}
         															     parametros_llamadas.hident := array_o_llamada.hident;
       																     parametros_llamadas.hrequireFunc := true;
                    									\end{lstlisting} \\
							& 					& parametros\_llamadas \\ 
							&					& \begin{lstlisting}
            																ADD_INST ( "call " || array_o_llamada.hident );
            																acceso_a_array_opcional.hnombre := NEW_IDENT();
            																acceso_a_array_opcional.htipo := parametros_llamadas.tipoRetorno;
            																ADD_INST ( "store_function_ret " || acceso_a_array_opcional.hnombre );
                    								\end{lstlisting} \\
							& 				 	& acceso\_a\_array\_opcional \\
							&					& \begin{lstlisting}
            																array_o_llamada.nombre := acceso_a_array_opcional.nombre;
            																array_o_llamada.tipo := acceso_a_array_opcional.tipo;
                    									\end{lstlisting} \\
									
							& |				 	& \begin{lstlisting}
            																acceso_a_array.hident := array_o_llamada.hident;
            																acceso_a_array.htipo := NEW_BASIC_TYPE(UNKNOWN);
                    									\end{lstlisting} \\
							&					& acceso\_a\_array \\
							&					& \begin{lstlisting}
            																array_o_llamada.nombre := NEW_IDENT ();
            																ADD_INST ( array_o_llamada.nombre || " := " || array_o_llamada.hident
																		¡\hspace{5em}¡ || "[" || acceso_a_array.offset || "]" );
            																array_o_llamada.tipo := acceso_a_array.tipo;
                    									\end{lstlisting} \\

							& |				 	& $\xi$ \\
							&					& \begin{lstlisting}
        array_o_llamada.nombre := array_o_llamada.hident;

        if ( ST_EXISTS(array_o_llamada.hident) == false )
        {
            ERROR( "Unknow identifier " || array_o_llamada.hident );
            array_o_llamada.tipo := NEW_BASIC_TYPE ( UNKNOWN );
        }
        else
        {
            array_o_llamada.tipo := ST_GET_TYPE ( array_o_llamada.hident );
        }
                    									\end{lstlisting} \\

\end{tabular}


\small
\begin{tabular}{r c p{.72\textwidth}}

		\espacio
		
			acceso\_a\_array\_opcional	& $\longrightarrow$ 	& \begin{lstlisting}
																acceso_a_array.hident := "`function return value`";
																acceso_a_array.htipo := acceso_a_array_opcional.htipo;
                    									\end{lstlisting} \\
										& 					& acceso\_a\_array \\
										&					& \begin{lstlisting}
            acceso_a_array_opcional.nombre := NEW_IDENT ();
            ADD_INST ( acceso_a_array_opcional.nombre || " := " || acceso_a_array_opcional.hnombre || "[" || acceso_a_array.offset || "]" );
            acceso_a_array_opcional.tipo := acceso_a_array.tipo;
                    									\end{lstlisting} \\
									
										& | 					& $\xi$ \\
										&					& \begin{lstlisting}
        acceso_a_array_opcional.nombre := acceso_a_array_opcional.hnombre;
        acceso_a_array_opcional.tipo := acceso_a_array_opcional.htipo;
                    									\end{lstlisting} \\
										

			lista\_de\_expr	& $\longrightarrow$ 	& expresion resto\_lista\_expr \\
							&					& \begin{lstlisting}
        lista_de_expr.exprs := JOIN(INIT_LIST(expresion.nombre), resto_lista_expr.exprs);
        lista_de_expr.tipos := JOIN(INIT_LIST(expresion.tipo), resto_lista_expr.tipos);
        lista_de_expr.literales := JOIN(INIT_LIST(expresion.literal), resto_lista_expr.literales);
                    									\end{lstlisting} \\
								& | 					& $\xi$ \\
								&					& \begin{lstlisting}
        lista_de_expr.exprs := EMPTY_LIST();
        lista_de_expr.tipos := EMPTY_LIST();
        lista_de_expr.literales := EMPTY_LIST();
                    									\end{lstlisting} \\
																		
			resto\_lista\_expr	& $\longrightarrow$ 	& \ter{,} expresion resto\_lista\_expr \\
								&					& \begin{lstlisting}
        resto_lista_expr.exprs := JOIN(INIT_LIST(expresion.nombre), resto_lista_expr$_{1}$.exprs);
        resto_lista_expr.tipos := JOIN(INIT_LIST(expresion.tipo), resto_lista_expr$_{1}$.tipos);
        resto_lista_expr.literales := JOIN(INIT_LIST(expresion.literal), resto_lista_expr$_{1}$.literales);
                    									\end{lstlisting} \\
									
								& | 					& $\xi$ \\
								&					& \begin{lstlisting}
        resto_lista_expr.exprs := EMPTY_LIST();
        resto_lista_expr.tipos := EMPTY_LIST();
        resto_lista_expr.literales := EMPTY_LIST();
                    									\end{lstlisting} \\
        
        
\end{tabular}


\small
\begin{tabular}{r c p{.72\textwidth}}

		\espacio
		
			opl1							& $\longrightarrow$ 	& \ter{*} \\
										&					& \begin{lstlisting}
																opl1.op := "*";
                    											  \end{lstlisting} \\
										& |					& \ter{/} \\
										&					& \begin{lstlisting}
																opl1.op := "/";
															  \end{lstlisting} \\

			opl2							& $\longrightarrow$ 	& \ter{+} \\
										&					& \begin{lstlisting}
																opl2.op := "+";
                    											  \end{lstlisting} \\
										& |					& \ter{-} \\
										&					& \begin{lstlisting}
																opl2.op := "-";
															  \end{lstlisting} \\        
			booleano						& $\longrightarrow$ 	& \ter{true} \\
										&					& \begin{lstlisting}
																booleano.value := "true";
                    											  \end{lstlisting} \\
										& |					& \ter{false} \\
										&					& \begin{lstlisting}
																booleano.value := "true";
															  \end{lstlisting} \\
															     
			M							& $\longrightarrow$ 	& $\xi$ \\
										&					& \begin{lstlisting}
																M.ref := GET_REF();
                    											  \end{lstlisting} \\
											  
\end{tabular}


\small
\begin{tabular}{r c p{.72\textwidth}}
	  
			oprel						& $\longrightarrow$ 	& \ter{<} \\
										&					& \begin{lstlisting}
																oprel.op := "<";
                    											  \end{lstlisting} \\
										& |					& \ter{>} \\
										&					& \begin{lstlisting}
																oprel.op := ">";
															  \end{lstlisting} \\
  										& |					& \ter{<=} \\
										&					& \begin{lstlisting}
																oprel.op := "<=";
															  \end{lstlisting} \\
  										& |					& \ter{>=} \\
										&					& \begin{lstlisting}
																oprel.op := ">=";
															  \end{lstlisting} \\
  										& |					& \ter{==} \\
										&					& \begin{lstlisting}
																oprel.op := "==";
															  \end{lstlisting} \\          
  										& |					& \ter{/=} \\
										&					& \begin{lstlisting}
																oprel.op := "/=";
															  \end{lstlisting} \\ 															  
\end{tabular}
